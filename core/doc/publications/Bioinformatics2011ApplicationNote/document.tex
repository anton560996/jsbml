\documentclass{bioinfo}
\copyrightyear{2011}
\pubyear{2011}

\usepackage{subfig}
\usepackage{listings}
\usepackage{ifthen}

% \usepackage{ulem}

% some nice colors
\definecolor{royalblue}{cmyk}{.93, .79, 0, 0}
\definecolor{lightblue}{cmyk}{.10, .017, 0, 0}
\definecolor{forrestgreen}{cmyk}{.76, 0, .76, .45}
\definecolor{darkred}{rgb}{.7,0,0}
\definecolor{winered}{cmyk}{0,1,0.331,0.502}
\definecolor{lightgray}{gray}{0.97}

\lstset{language=Java,
morendkeywords={String, Throwable}
captionpos=b,
basicstyle=\scriptsize\ttfamily,%\bfseries
stringstyle=\color{darkred}\scriptsize\ttfamily,
keywordstyle=\color{royalblue}\bfseries\ttfamily,
ndkeywordstyle=\color{forrestgreen},
numbers=left,
numberstyle=\scriptsize,
% backgroundcolor=\color{lightgray},
breaklines=true,
tabsize=2,
frame=single,
breakatwhitespace=true,
identifierstyle=\color{black},
% morecomment=[l][\color{forrestgreen}]{//},
% morecomment=[s][\color{lightblue}]{/**}{*/},
% morecomment=[s][\color{forrestgreen}]{/*}{*/},
commentstyle=\ttfamily\itshape\color{forrestgreen}
% framexleftmargin=5mm,
% rulesepcolor=\color{lightgray}
% frameround=ttff
}

\hyphenation{
SBML-squeez-er
Tree-Node
}


\begin{document}
\firstpage{1}
\application
\title[JSBML: The Java library for SBML]{JSBML: a flexible Java library
for working with SBML} \author[Dr\"ager and Rodriguez \textit{et~al.}]{Andreas
Dr\"ager\,$^{1,*}$,
Nicolas Rodriguez\,$^{2,*}$,
Marine Dumousseau\,$^{2}$,
Alexander~D\"orr\,$^{1}$,
Clemens Wrzodek\,$^{1}$,
Nicolas Le Nov\`{e}re\,$^{2}$,
Andreas Zell\,$^{1}$,
Michael~Hucka\,$^{3,}$\footnote{to whom correspondence should be addressed}}
\address{$^{1}$Center for Bioinformatics Tuebingen (ZBIT), University of
Tuebingen, T\"ubingen, Germany\\
$^{2}$European Bioinformatics Institute, Wellcome Trust Genome Campus,
Hinxton, Cambridge, UK\\
$^{3}$Computing and Mathematical Sciences, California Institute of Technology,
Pasadena, CA, USA}

\history{Received on XXXXX; revised on XXXXX; accepted on XXXXX}

\editor{Associate Editor: XXXXXXX}

\maketitle


\begin{abstract}

\section{Summary:}
The specifications of the Systems Biology Markup Language (SBML) define
standards for storing and exchanging computer models of biological processes in
text files. In order to perform model simulations, graphical visualizations, and
other software manipulations, an in-memory representation of SBML is required. We
developed JSBML for this purpose. In contrast to prior implementations of SBML
APIs, JSBML has been designed from the ground up for the Java\texttrademark{}
programming language, and can therefore be used on all platforms supported by a
Java Runtime Environment. This offers important benefits for Java users,
including the ability to distribute software as Java Web Start applications.
JSBML supports all SBML Levels and Versions through Level~3 Version~1, and we
have strived to maintain the highest possible degree of compatibility with the
popular library libSBML. JSBML also supports modules that can facilitate the
development of plugins for end-user applications, as well as ease migration from
a libSBML-based backend.

\section{Availability:} Source code, binaries, and documentation for JSBML can
be freely obtained under the terms of the LGPL 2.1 from the website
\href{http://sbml.org/Software/JSBML}{http://sbml.org/Software/JSBML}.

\section{Contact:} \href{mailto:jsbml-team@sbml.org}{jsbml-team@sbml.org}

\section{Supplementary information:} Supplementary data are available at
Bioinformatics online.

\end{abstract}


\section{Introduction}

The XML-based Systems Biology Markup Language (SBML, \citealt{M.Hucka03012003})
is the \emph{de facto} standard file format for the storage and exchange of
quantitative computational models in systems biology, supported by more than
210 software packages to date (Mar.~2011).
Much of this success is due to its
clearly defined specifications and the availability of libSBML~\citep{Bornstein2008},
a portable, robust, full-featured, and easy-to-use library.

LibSBML provides many methods for the manipulation and validation of
SBML files through its Application Programming Interface (API).
Primarily written in C and C++, libSBML also provides automatically-generated
language bindings for Java\texttrademark, among other programming languages.
%, MATLAB\texttrademark, Perl, and many more}.
However, the full platform independence brought by the use of Java is limited in
libSBML because the binding is only a wrapper
around the C/C++ core, implemented using the Java Native Interface (JNI).
As a consequence, some software developers experience difficulties
deploying portable libSBML-based Java applications.
% Furthermore, the libSBML API and type hierarchy are not sufficiently
% intuitive from a Java programmer's perspective just because they were not
% designed directly for Java.
Several groups in the SBML community thus began to
develop their own Java libraries for SBML. To avoid needless duplication, some of
these groups recently pooled their efforts and created JSBML, an
open-source project to develop a pure Java library for SBML.

The primary aim of the project is to provide an API 
that maps all SBML elements to a flexible and extended
Java type hierarchy. Where possible, JSBML strives for
100\,\% compatibility with libSBML's Java API, to ease the transition from
one library to the other. There are currently no plans to re-implement 
the more complex functionalities of libSBML, such as model consistency checking,
SBML validation, and conversion between different SBML Levels and Versions;
separate community efforts are underway to provide such libSBML facilities
via web services.

The software produced by the project is freely available from
\href{http://sbml.org/Software/JSBML}{http://sbml.org/Software/JSBML}.


%\section{Approach}

%\begin{methods}
%\section{Methods}
\section{Implementation}

A key achievement of the JSBML project is the development of an extended type
hierarchy, designed from scratch based on the SBML specifications, but still
following the naming conventions of methods and classes in
libSBML. For each element defined in at least one SBML Level/Version combination,
JSBML provides a corresponding class reflecting all of its properties. SBML
elements or attributes not part of higher SBML Levels (removed or made
obsolete) are marked as deprecated. JSBML defines superclasses or interfaces for
elements that share common properties. For instance, the interface
\texttt{NamedSBase} does not directly correspond to a data type in one of the
SBML specifications, but serves as the superclass of all \texttt{SBase}-derived
classes that can be addressed by an identifier and a name. Similarly, all classes
that may contain a mathematical expression implement the
interface \texttt{MathContainer}. A full overview of this type hierarchy can be
found in the supplementary data associated with this article. JSBML also
supports SBML \emph{notes} in XHTML format, as well as SBML
\emph{annotations}, including MIRIAM identifiers \citep{Novere2005} and
SBO terms \citep{Novere2006b}.
When building JSBML, the latest SBO OBO export can directly be downloaded and
parsed \citep{Holland2008}.
% Predefined units are automatically annotated with the correct
% Unit Ontology terms.
The \texttt{Model} class provides several methods, all beginning with the name
\texttt{find*}, for querying SBML elements. Filters enable users to search lists
for elements that possess specific properties. All \texttt{ListOf*} elements in
JSBML implement Java's \texttt{List} interface, making iteration and the use of
generic Java types possible. Fig.~\ref{fig:JSBML} demonstrates how the
hierarchically structured content of an SBML file can be easily visualized in the form
of a tree.
\begin{figure*}
\centerline{
  \subfloat[The SBML parser in JSBML understands the hierarchical data structure
of SBML.]{\label{lst:JSBMLVisualizer}
    \raisebox{2.82cm}{
      \parbox{.55\textwidth}{
        \begin{minipage}[t][5.7cm][c]{.55\textwidth}
          \lstinputlisting[language=Java]{src/JSBMLvisualizer.java}
        \end{minipage}
      }
    }
  }\hspace{1cm}
  \subfloat[Example for SBML test case 26.]{\label{fig:JSBMLVisualizer}
%\raisebox{1.2cm}{
\includegraphics[width=.24\textwidth,height=170pt]{img/Case26_Tree_Windows.png}
%}
  }
}
\caption[Using JSBML for reading and visualizing an SBML file using
JSBML]{Using JSBML for reading and visualizing an SBML file. The type
\texttt{SBase} extends the Java interfaces \texttt{Serializable} for saving
JSBML objects in binary form or sending them over a network connection,
\texttt{Cloneable} for creating deep object copies, and \texttt{TreeNode}.
The last interface allows callers to apply any recursive operation, such as
using \texttt{JTree} for display (see~\ref{fig:JSBMLVisualizer} for an
example).}
\label{fig:JSBML}
\end{figure*}

JSBML includes parsers that read mathematical formulas in both MathML
format and an infix formula syntax.  Internally, it converts formulas
into an abstract syntax tree representation; it can write out the
trees in MathML, infix, and \LaTeX{} formula notations.
% and also be visualized in a \texttt{JTree} since they also implement
% \texttt{TreeNode}.
In addition, although JSBML does not implement full-featured
consistency checking of SBML models, it does throw Java exceptions in some
situations to prevent users from creating invalid content.
It implements a check for overdetermined models
using the algorithm of \citet{Hopcroft1973}; this is 
also used to identify variables in algebraic rules. Further, JSBML can
automatically derive the units of a mathematical expression.
Whenever a property of some \texttt{SBase} is altered, an
\texttt{SBaseChangeEvent} is fired that notifies dedicated listeners.
As one possible application, graphical user interfaces could automatically
react when the model is changed. Using modules, JSBML capabilities can be
further extended; it can therefore be used as a communication layer between an
application and libSBML or CellDesigner \citep{Funahashi2003}---this also
facilitates turning an existing application into a plugin for CellDesigner.
%\end{methods}

%\section{Implementation}

% JSBML is written entirely in Java version 1.5 and does
% not require additional non-Java software.
% % It was successfully tested under MacOS X, Windows 7, Vista, XP, and Linux
% % (openSuSE 11.2, Ubuntu 10.04).
% It is distributed in source-code form as well as precompiled JAR files,
% with different JAR distributions available depending on whether required
% third-party libraries are included. We also provide a convenient build
% file offering several options for users to easily create customized JAR files.
% Since it is distributed under the terms of the Lesser GNU Public License (LGPL), 
% JSBML can freely be used even in proprietary software.

%\section{Discussion}


\section{Conclusion}

JSBML is an ongoing project that provides comprehensive and entirely Java-based
data structures to read, write, and manipulate SBML files. Its layered
architecture allows for the creation of Java Web Start applications and
CellDesigner plugins based on stand-alone programs with very little effort.
%
New versions of SBMLsqueezer \citep{Draeger2008} and Biomodels Database
\citep{Li2010} have  already been released using JSBML. Many other projects are
planned.

\section*{Acknowledgement}

\paragraph{Authors' contribution\textcolon} NR and AD contributed equally to
this work.

\paragraph{Funding\textcolon}
The development of JSBML is funded by a grant from the National Institute of
General Medical Sciences (NIGMS, USA), funds from EMBL-EBI (Germany, UK), and
the Federal Ministry of Education and Research (BMBF, Germany) in the projects
Virtual Liver and Spher4Sys (grant numbers 0315756 and 0315384C).

\paragraph{Conflict of Interest\textcolon} none declared.

%\bibliographystyle{natbib}
%\bibliographystyle{achemnat}
%\bibliographystyle{plainnat}
%\bibliographystyle{abbrv}
%\bibliographystyle{bioinformatics}
%
%\bibliographystyle{plain}
%
%\bibliography{Document}


% \bibliographystyle{natbib}
% \bibliography{../../literature}

\begin{thebibliography}{}

\bibitem[Bornstein {\em et~al.}(2008)Bornstein, Keating, Jouraku, and
  Hucka]{Bornstein2008}
Bornstein {\em et~al.} (2008).
\newblock {LibSBML: an API Library for SBML}.
\newblock {\em Bioinformatics\/}, {\bf 24}(6), 880--881.

\bibitem[Dr{\"a}ger {\em et~al.}(2008)Dr{\"a}ger, Hassis, Supper, Schr{\"o}der,
  and Zell]{Draeger2008}
Dr{\"a}ger {\em et~al.} (2008).
\newblock {SBMLsqueezer: a CellDesigner plug-in to generate kinetic rate
  equations for biochemical networks}.
\newblock {\em BMC Syst. Biol.\/}, {\bf 2}(1), 39.

% \bibitem[Dr{\"a}ger {\em et~al.}(2009)Dr{\"a}ger, Planatscher, Wouamba,
%   Schr{\"o}der, Hucka, Endler, Golebiewski, M{\"u}ller, and Zell]{Draeger2009}
% Dr{\"a}ger {\em et~al.} (2009).
% \newblock {{SBML2\LaTeX: Conversion of SBML files into human-readable
%   reports}}.
% \newblock {\em Bioinformatics\/}, {\bf 25}(11), 1455--1456.

\bibitem[Funahashi {\em et~al.}(2003)Funahashi, Tanimura, Morohashi, and
  Kitano]{Funahashi2003}
Funahashi {\em et~al.} (2003).
\newblock {CellDesigner: a process diagram editor for gene-regulatory and
  biochemical networks}.
\newblock {\em BioSilico\/}, {\bf 1}(5), 159--162.

\bibitem[Holland {\em et~al.}(2008)Holland, Down, Pocock, Prli\'{c}, Huen,
  James, Foisy, Dr\"ager, Yates, Heuer, and Schreiber]{Holland2008}
Holland {\em et~al.} (2008).
\newblock {BioJava: an Open-Source Framework for Bioinformatics}.
\newblock {\em Bioinformatics\/}, {\bf 24}(18), 2096--2097.

\bibitem[Hopcroft and Karp(1973)Hopcroft and Karp]{Hopcroft1973}
Hopcroft and Karp (1973).
\newblock {An $n^{5/2}$ algorithm for maximum matchings in bipartite graphs}.
\newblock {\em SIAM J. Comput.\/}, {\bf 2}, 225.

\bibitem[Hucka {\em et~al.}(2003)Hucka, Finney, Sauro, Bolouri, Doyle, Kitano,
  Arkin, Bornstein, Bray, Cornish-Bowden, Cuellar, Dronov, Gilles, Ginkel, Gor,
  Goryanin, Hedley, Hodgman, Hofmeyr, Hunter, Juty, Kasberger, Kremling,
  Kummer, Le~Nov{\`e}re, Loew, Lucio, Mendes, Minch, Mjolsness, Nakayama,
  Nelson, Nielsen, Sakurada, Schaff, Shapiro, Shimizu, Spence, Stelling,
  Takahashi, Tomita, Wagner, Wang, and the rest of~the
  SBML~Forum]{M.Hucka03012003}
Hucka {\em et~al.} (2003).
\newblock {The systems biology markup language (SBML): a medium for
  representation and exchange of biochemical network models}.
\newblock {\em Bioinformatics\/}, {\bf 19}(4), 524--531.

\bibitem[Le~Nov{\`e}re {\em et~al.}(2005)Le~Nov{\`e}re, Finney, Hucka, Bhalla,
  Campagne, Collado-Vides, Crampin, Halstead, Klipp, Mendes, Nielsen, Sauro,
  Shapiro, Snoep, Spence, and Wanner]{Novere2005}
Le~Nov{\`e}re {\em et~al.} (2005).
\newblock {Minimum information requested in the annotation of biochemical
  models (MIRIAM)}.
\newblock {\em Nat. Biotechnol.\/}, {\bf 23}(12), 1509--1515.

\bibitem[Le~Nov{\`e}re {\em et~al.}(2006)Le~Nov{\`e}re, Courtot, and
  Laibe]{Novere2006b}
Le~Nov{\`e}re {\em et~al.} (2006).
\newblock Adding semantics in kinetics models of biochemical pathways.
\newblock In Kettner and Hicks, eds., {\em 2\textsuperscript{nd}
  International ESCEC Workshop.
%   on Experimental Standard Conditions on Enzyme
%   Characterizations.
  Beilstein Institut, R{\"u}desheim, Germany\/}, pages
  137--153, R{\"u}dessheim/Rhein, Germany. ESEC.

\bibitem[Li {\em et~al.}(2010)Li, Donizelli, Rodriguez, Dharuri, Endler, Chelliah, Li, 
He, Henry, Stefan, Snoep, Hucka, Le Nov{\`e}re and Laibe]{Li2010}
Li {\em et~al.} (2010).
\newblock BioModels Database: An enhanced, curated and annotated resource for published quantitative kinetic models.
\newblock {\em BMC Syst Biol}, 2010, 4, 92
%   title   = {{BioModels Database: An enhanced, curated and annotated resource for published quantitative kinetic models.}},
%   journal = {BMC Systems Biology},
%   year    = {2010},
%   month   = {Jun},
%   volume  = {4},
%   pages   = {92},
%   pmid    = {20587024}




\end{thebibliography}

\end{document}
