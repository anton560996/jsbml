% -*- TeX-master: "User_guide"; fill-column: 75 -*-

\chapter{Implementing extensions in JSBML}
\label{sec:howToExtension}

In this chapter, we describe how to get started with writing an extension
for JSBML to support an \SBMLthree package.  We use a concrete (though
artificial) example to illustrate various points.  This example extension
is named, very cleverly, \emph{Example}, and while it does not actually do
anything significant, we hope it will help make the explanations more
understandable.


\section{Organizing the source code}
\label{sec:example-organization}

In the JSBML SVN repository, all extensions are found in the subdirectory
named \code{extensions} inside the \code{trunk} directory.  (The process
for checking out a local copy of the repository is described
in \vref{sec:SourceDistribution}.) Each extension is named after the
corresponding SBML short name for the \SBMLthree package; for example,
\code{fbc} for the Flux Balance Constraints package, \code{layout} for the
Layout package, and so on.  The source directories for the extensions
follow some basic conventions for their organization and contents.

\begin{wrapfigure}[23]{r}{2.31in}
  \vspace*{-2ex}
  \dirtree{%
    .1 /.
    .2 build.xml.
    .2 doc/.
    .3 img/.
    .2 lib/.
    .2 resources/.
    .2 src/.
    .3 org/.
    .4 sbml/.
    .5 jsbml/.
    .6 ext/.
    .7 NAME/.
    .6 xml/.
    .7 parsers/.
    .2 test/.
  }
  \caption{Typical structure of the source directory for a JSBML extension.
    The root of the tree shown here is the \code{extensions/NAME} subdirectory,
    which is located within the \code{trunk} subdirectory of the JSBML SVN
    repository.}
  \label{fig:extension-src-structure}
\end{wrapfigure}
As part of creating a new extension for JSBML, please follow the same
conventions as those used in the existing extension directories.  These
conventions are illustrated in \vref{fig:extension-src-structure}.
There should be a build script in a file named ``\code{build.xml}'' for use
with Ant~\citep{ApacheAnt}, and several subdirectories.  The \code{doc}
subdirectory should contain documentation about the extension, preferrably
with a subdirectory of its own, \code{img}, containing a UML diagram of the
type hierarchy of the package.  This can be in the form of, for instance, a
Graphviz~\cite{graphvizWebsite} file \code{type\_hierarchy.dot}, so that
the diagram can be generated in different image formats.  The extension
directory should also contain a \code{lib} subdirectory where any
package-specific, third-party libraries are located; a \code{resources}
subdirectory for any non-source files that may be required by the extension
code; an \code{src} subdirectory for the Java source code comprising the
extension; and finally, a \code{test} subdirectory containing tests for the
extension code, preferrably in JUnit~\cite{junitWebsite} format.

Note the structure of the \code{src} subdirectory. A JSBML extension needs
to define at least two packages: \code{org.sbml.jsbml.ext.NAME}, for the
data structures and code for defining and manipulating the SBML components
specified by the extension, and \code{org.sbml.jsbml.xml.parsers}, for the
parsing code for reading and writing SBML files with the extension
constructs.  As per Java conventions, these source subdirectories are
organized hierarchically based on the package components, which leads to
the nested structure shown in \vref{fig:extension-src-structure}.


\section{Creating the object hierarchy}

A JSBML extension may need to do different things depending on the details
of the \SBMLthree package that it implements.  In this section, we discuss
various common actions and how they can be written in JSBML.


\subsection{Introducing new components and extending others}
\label{subsec:addingClasses}

Most \SBMLthree packages extend existing SBML core components or define
entirely new components.  A common need for packages is to extend the SBML
\Model object, so we begin by explaining how this can be achieved.

\begin{figure}[t]
  \begin{example}[numbers=left]
public class ExampleModel extends AbstractSBasePlugin {

  // Basic constructor.
  public ExampleModel(Model model) {
    super(model);
  }

  // Returns the model.
  public Model getModel() {
    return (Model) getExtendedSBase;
  }
}\end{example}
  \caption{How to extending \AbstractSBasePlugin to create an extended
    \,\Model.}
  \label{lst:ModelExtClass}
\end{figure}

\Vref*{lst:ModelExtClass} shows the beginnings of the definition for
a class named \code{ExampleModel} that extends the plain SBML
\Model. Technically, an extension really only needs to implement the
\SBasePlugin interface, but because the abstract class \AbstractSBasePlugin
implements important and useful methods, it is generally preferrable to
extend that one instead.  In this example, our constructor for
\code{ExampleModel} accepts an object that is a \Model, because that is
what we want to extend.  The call to the super constructor will save the
given model as the \SBase object that is being extended, and it will store
it in an attribute named \code{extendedSBase}. Our example
\code{ExampleModel} class also adds a method, \code{getModel()}, to
retrieve the extended model object.

In most cases, extensions will also introduce new components that have no
counterpart in the SBML core.  We illustrate this here by creating a
component called \code{Foo} with three attributes: \code{id}, \code{name},
and an integer-valued attribute, \code{bar}.  We assume that in the
(hypothetical) package specification for \emph{Example}, \code{Foo} is
derived from \SBase; let us also assume that \emph{Example} provides the
ability to attach a list of \code{Foo} objects to an extended version of
\Model.  We show in \vref{sec:listofs} how to create the list
structure; here, we focus on the definition of \code{Foo}.  We define the
class \code{Foo} by extending \AbstractSBasePlugin, and add methods for
working with the attributes.  In \vref{lst:DefaultModelExtFooBar},
we list the code so far, focusing on just one of the attributes,
\emph{bar}.

\begin{figure}[hbt]
  \begin{example}[numbers=left]
// Use Integer, so we can denote unset values as null public Integer bar;

public int getBar() {
  if (isSetBar()) {
    return bar.intValue();
  }
  // This is necessary because we cannot return null here.
  throw new PropertyUndefinedError(ExampleConstant.bar, this);
}

public boolean isSetBar() {
  return this.bar != null;
}

public void setBar(int value) {
  Integer oldBar = this.bar;
  this.bar = bar;
  firePropertyChange(ExampleConstant.bar, oldBar, this.bar);
}

public boolean unsetBar() {
  if (isSetBar()) {
    Integer oldBar = this.bar;
    this.bar = null;
    firePropertyChange(ExampleConstant.bar, oldBar, this.bar);
    return true;
  }
  return false;
}\end{example}
  \caption{Implementation of the five necessary methods that should be
    created for every attribute on class \code{Foo}.  Note: if attribute
    \emph{bar} had been a boolean-valued attribute, we would also provide
    the method \code{isBar()}, whose implementation would delegate to
    \code{getBar()}.}
  \label{lst:DefaultModelExtFooBar}
\end{figure}

A few points about the code of \vref{lst:DefaultModelExtFooBar} are
worth mentioning.  The identifiers on SBML components are often required to
be unique; for many components, the scope of uniqueness is the entire set
of main SBML components (e.g., \Species, \Compartment{}s, etc.), but some
have uniqueness requirements that are limited to some subset of entities
(e.g., unit names).  For the purposes of this example, we assume that the
identifiers of \emph{Foo} objects in a model must be unique across all
identifiers in the model.  All entities that have such uniqueness
constraints should implement the JSBML interface \UniqueNamedSBase; in our
example, this is taken care of by the abstract superclasses, so nothing
needs to be done explicitly here.

The code in \vref{lst:DefaultModelExtFooBar} also illustrates
another point, the need call to \code{firePropertyChange()} in set and
unset methods.  This is needed in order to ensure that all listeners are
notified about changes to the objects.  Finally, note that in cases that
the return type is a Java base type, such as \code{int} or \code{boolean},
but the corresponding internal element (e.g., \code{Integer} or
\code{Boolean}) is set to \code{null}, the program must throw 
\code{PropertyUndefineError} in the get method to prevent incorrect
results (see line~8).

The last basic matter that needs to be addressed is the definition of
appropriate class constructors for our class \code{Foo}.  The minimum we
need to define is a constructor that takes no arguments.  Even though some
or all of the attributes of a class may be mandatory, default constructors
that take no arguments still need to be defined in JSBML.  This is due to
the internal working of parsers that read SBML files and create the data
structure in memory.  All attributes can be set after the object has been
created.

\begin{figure}[b]
  \begin{example}[numbers=left]
public Foo() {
  super();
  initDefaults();
}

public Foo(String id) {
  super(id);
  initDefaults();
}

public Foo(int level, int version){
  this(null, null, level, version);
}

public Foo(String id, int level, int version) {
  this(id, null, level, version);
}

public Foo(String id, String name, int level, int version) {
  super(id, name, level, version);
  if (getLevelAndVersion().compareTo(Integer.valueOf(3), Integer.valueOf(1)) < 0) {
    throw new LevelVersionError(getElementName(), level, version);
  }
  initDefaults();
}

/** Clone constructor */
public Foo(Foo foo) {
  super(foo);
  bar = foo.bar;
}

public void initDefaults() {
  addNamespace(ExampleConstant.namespaceURI);
  bar = null;
}\end{example}
  \caption{Constructors for class \code{Foo}.  Note the code testing for
    the SBML Level and Version, on lines~21--23; since this extension
    implements a hypothetical package for \SBMLthree, the code here rejects
    anything before Level~3 Version~1 by throwing the JSBML exception
    \LevelVersionError.}
  \label{lst:ModelExtFooConstructors}
\end{figure}

Beyond this, the precise combination of constructor arguments defined for a
class is a design issue that must be decided for each class individually.
Attempting to define a separate constructor for every possible combination
of arguments can lead to a combinatorial explosion, resulting in complex
class definitions, confusing code, and excessive maintenance costs, so it
is better to decide which combinations of arguments are the most common and
focus on them.  In \vref{lst:ModelExtFooConstructors}, we show a
recommended selection of constructors.  They include a constructor that
takes an identifier onely, another that takes SBML Level and Version values
only, and another that takes all arguments together.  If you delegate the
constructor call to the super class, you have to take care of the
initialization of your custom attributes yourself (by calling a method like
\code{initDefaults()}).  If you delegate to another constructor in your
class, you only have to do that at the last one in the delegation chain.


\subsection{\codeNC{ListOf}s}
\label{sec:listofs}

Our hypothetical \emph{Example} extension adds no new attributes to the
extended \Model itself, but it does introduce the ability to have a list of
\code{Foo} objects as a child of \Model.  In JSBML, this will be
implemented using Java generics and the class \code{ListOf}, such that the
type of the list will be \code{ListOf<Foo>}.  (Unlike in libSBML, there
will not be an actual separate \code{ListOfFoo} class.)  In
\vref{lst:ModelExtListOfFoosBasic}, we show the basic implementation
of the methods that would be added to \Model to handle
\code{ListOf<Foo>}: \code{isSetListOfFoos()}, \code{getListOfFoos()},
\code{setListOfFoos(ListOf<Foo>)}, and \code{unsetListOfFoos()}.

\begin{figure}[b]
  \begin{example}[numbers=left]
public boolean isSetListOfFoos() {
  if ((listOfFoos == null) || listOfFoos.isEmpty()) {
    return false;
  }
  return true;
}

public ListOf<Foo> getListOfFoos() {
  if (!isSetListOfFoos()) {
    Model m = getModel();
    listOfFoos = new ListOf<Foo>(m.getLevel(), m.getVersion());
    listOfFoos.addNamespace(ExampleConstants.namespaceURI);
    listOfFoos.setSBaseListType(ListOf.Type.other);
    m.registerChild(listOfFoos);
  }
  return ListOfFoos;
}

public void setListOfFoos(ListOf<Foo> listOfFoos) {
  unsetListOfFoos();
  this.listOfFoos = listOfFoos;
  getModel().registerChild(this.listOfFoos);
}

public boolean unsetListOfFoos() {
  if(isSetListOfFoos()) {
    ListOf<Foos> oldFoos = this.listOfFoos;
    this.listOfFoos = null;
    oldFoos.fireNodeRemovedEvent();
    return true;
  }
  return false;
}\end{example}
  \caption{Implementation of the methods \code{isSetListOfFoos()},
    \code{getListOfFoos()}, and \code{setListOfFoos()}.}
  \label{lst:ModelExtListOfFoosBasic}
\end{figure}

Typically, when adding and removing Foo objects to the \Model, direct
access to the actual \code{ListOf} object is not necessary.  To add and
remove \code{Foo} objects from a given SBML model, it is more convenient to
have methods to add and remove on \code{Foo} object at a time.  We show
such methods in \vref{lst:ModelExtAddRemoveFoos}.  The methods also
do some additional consistency checking as part of their work.

\begin{figure}[t]
  \begin{example}[numbers=left]
public boolean addFoo(Foo foo) {
    return getListOfFoos().add(foo);
}

public boolean removeFoo(Foo foo) {
  if (isSetListOfFoos()) {
    return getListOfFoos().remove(foo);
  }
  return false;
}

public void removeFoo(int i) {
  if (!isSetListOfFoos()) {
    throw new IndexOutOfBoundsException(Integer.toString(i));
  }
  listOfFoos.remove(i);
}

// If the ID is mandatory for Foo objects, one should also add the following:
public void removeFoo(String id) {
  return getListOfFoos().removeFirst(new NameFilter(id));
}\end{example}
  \caption{Implementation of \code{ListOf} methods \code{addFoo(Foo foo)},
    \code{removeFoo(Foo foo)}, \code{removeFoo(int i)}.}
  \label{lst:ModelExtAddRemoveFoos}
\end{figure}

\begin{figure}[hb]
  \begin{example}[numbers=left]
public boolean getAllowsChildren() {
  return true;
}

public int getChildCount() {
  int count = 0;
  if (isSetListOfFoos())
    count++;
  return count;  // same for each additional ListOf* in this extension
}

public SBase getChildAt(int childIndex) {
  if (childIndex < 0) {
    throw new IndexOutOfBoundsException(childIndex + " < 0");
  }

  int pos = 0;
  if (isSetListOfFoos()) {
    if (pos == childIndex)
      return getListOfFoos();
    pos++;
  }
  // same for each additional ListOf* in this extension
  throw new IndexOutOfBoundsException(MessageFormat.format(
    "Index {0,number,integer} >= {1,number,integer}", childIndex, +((int) Math.min(pos, 0))));
}\end{example}
  \caption{Methods which need to be implemented to make the children
    available in the extended model.}
  \label{lst:ModelExtChildren}
\end{figure}

To let a \code{ListOfFoo} appear as a child of the standard \Model, some
important methods for \TreeNode need to be implemented (see
\vref{lst:ModelExtChildren}).  Method \code{getAllowsChildren()} should
return \code{true} in this case, since this extension allows children.  The
child count and the indices of the children is a bit more complicated,
because they vary with the number of non-empty \code{ListOf}s.  So, for
every non-empty \code{ListOf} child of our model extension, we increase the
counter by one.  (Note also that if callers access list entries by index
number, they will need to take into account the possibility that a given
object's index may shift.)


\subsection{Methods for creating new objects}

Since a newly created instance of type \code{Foo} is not part of the model
unless it is added to it, \code{create*} methods should be provided that
take care of all that (see \vref{lst:ModelExtCreateMethods}).
These create methods should be part of the model to which the \code{Foo}
instance should be added, in this case \code{ExampleModel}.

\begin{figure}[thb]
  \begin{example}[numbers=left]
public class ExampleModel extends AbstractSBasePlugin {

  // ...

  // only, if ID is not mandatory in Foo
  public Foo createFoo() {
    return createFoo(null);
  }

  public Foo createFoo(String id) {
    Foo foo = new Foo(id, getModel().getLevel(), getModel().getVersion());
    addFoo(foo);
    return foo;
  }

  public Foo createFoo(String id, int bar) {
    Foo foo = createFoo(id);
    foo.setBar(bar);
    return foo;
  }
}\end{example}
  \caption{Convenience method to create \code{Foo} objects.}
  \label{lst:ModelExtCreateMethods}
\end{figure}


\subsection{The methods \codeNC{equals}, \codeNC{hashCode}, and \codeNC{clone}}

There are three more methods which should be implemented in an extension
class: \code{equals}, \code{hashCode} and \code{clone}.  This is not
different than whne implementing any other Java class, but because mistakes
here can lead to bugs that are very hard to find, we describe the process.

Whenever two objects \code{o1} and \code{o2} should be regarded as equal,
i.e., all their attributes are equal, the \code{o1.equals(o2)} and the
symmetric case \code{o2.equals(o1)} must return \code{true}, and otherwise
\code{false}. The \code{hashCode} method has two purposes here: allow a
quick check if two objects might be equal, and provide hash values for hash
maps or hash sets and such. The relationship between \code{equals} and
\code{hashCode} is that whenever \code{o1} is equal to \code{o2}, their
hash codes must be the same. Vice versa, whenever their hash codes are
different, they cannot be equal.

\ref{lst:ModelExtEquals} and \vref{lst:ModelExtHashCode} are examples
of how to write these methods for the class \code{Foo} with the attribute
\code{bar}.  Since \code{equals} accepts general objects, it first needs to
check if the passed object is of the same class as the object it is called
on.  Luckily, this has been implemented in \AbstractTreeNode, the super
class of \AbstractSBase. Each class only checks the attributes it adds to
the super class when extending it, but not the \code{ListOf}s, because they
are automatically checked in the \AbstractTreeNode class, the super class
of \AbstractSBase.

\begin{figure}[htb]
  \begin{example}[numbers=left]
@Override
public boolean equals(Object object) {
  boolean equals = super.equals(object);    // recursively checks all children
  if (equals) {
    Foo foo = (Foo) object;
    equals &= foo.isSetBar() == isSetBar();
    if (equals && isSetBar()) {
      equals &= (foo.getBar().equals(getBar()));
    }
    // ...
    // further attributes
  }
  return equals;
}\end{example}
  \caption{Example of the \code{equals} method.}
  \label{lst:ModelExtEquals}
\end{figure}

\begin{figure}[htb]
  \begin{example}[numbers=left]
@Override 
public int hashCode() {
  final int prime = 491;
  int hashCode = super.hashCode();    // recursively checks all children
  if (isSetBar()) {
    hashCode += prime * getBar().hashCode();
  }
  // ...
  // further attributes

  return hashCode;
}\end{example}
 \caption{Example of the \code{hashCode} method. The variable \code{prime}
   should be a large prime number to  prevent collisions.}
 \label{lst:ModelExtHashCode}
\end{figure}

\ref{lst:ModelExtClone} and~\vref{lst:ModelExtCloneFoo} illustrates
implementations of \code{clone()} methods.  To clone an object, the call to
the \code{clone()} method is delegated to a constructor of that class that
takes an instance of itself as argument.  There, all the elements of the
class must be copied, which may require recursive cloning.

\begin{figure}[htb]
  \begin{example}[numbers=left]
public ExampleModel clone() {
  return new ExampleModel(this);
}

public ExampleModel(ExampleModel model) {
  super();
  // deep cloning of all elements:
  if (model.isSetListOfFoos()) {
    listOfFoos = model.listOfFoos.clone();
  }
}\end{example}
 \caption{Example of the \code{clone} method for the \code{ExampleModel} class.}
 \label{lst:ModelExtClone}
\end{figure}

\begin{figure}[htb]
  \begin{example}[numbers=left]
public Foo clone() {
  return new Foo(this);
}

public Foo(Foo f) {
  super();

  // Integer objects are immutable, so it is sufficient to copy the pointer
  bar = f.bar;
}\end{example}
  \caption{Example of the \code{clone} method for the \code{Foo} class.}
  \label{lst:ModelExtCloneFoo}
\end{figure}


\section{Implementing the parser and writer for an SBML package}

One last thing is missing to be able to properly read and write SBML files
using the new extension: a parser and a writer. An easy way to do that is
to extend the \AbstractReaderWriter and implement the required methods. To
implement the parser, in this case the \code{ExamplePaser}, one should
start with two members and two simple methods, as shown in
\vref{lst:ModelExtParserClass}.


\subsection{Reading}

As can be seen from this code snippet, an additional class
\code{ExampleConstant} and an enum \code{ExampleList} are used.

\emph{\textbf{... TODO ...}}

\begin{figure}[htb]
  \begin{example}[numbers=left]
public class ExampleParser extends AbstractReaderWriter {

  /**
   * The logger for this parser
   */
  private Logger logger = Logger.getLogger(ExampleParser.class);

  /**
   * The ExampleList enum which represents the name of the list this parser is
   * currently reading.
   */
  private ExampleList groupList = ExampleList.none;

  /* (non-Javadoc)
   * @see org.sbml.jsbml.xml.parsers.AbstractReaderWriter#getShortLabel()
   */
  public String getShortLabel() {
    return ExampleConstant.shortLabel;
  }

  /* (non-Javadoc)
   * @see org.sbml.jsbml.xml.parsers.AbstractReaderWriter#getNamespaceURI()
   */
  public String getNamespaceURI() {
    return ExampleConstant.namespaceURI;
  }

}\end{example}
 \caption{The first part of the parser for the extension.}
 \label{lst:ModelExtParserClass}
\end{figure}


\subsection{Writing}

The method \code{getListOfSBMLElementsToWrite()} (see
\vref{lst:ModelExtParserListSBMLToWrite}) has to return a list of all
objects that have to be written because of the passed object.  In this way,
the writer can traverse the XML tree to write all nodes.  Basically, there
are three classes of objects that need to be distinguished:

\begin{itemize}
 \item \code{SBMLDocument}
 \item extended classes
 \item \code{TreeNode}
\end{itemize}

TODO: SBMLDocument.

After that we need to check if the current object is extendable using our extension.
In our example extension, a \Model{} can be extended using ExampleModel to also
have a list of Foos as children.
In Java, this ListOfFoos is not a children of the original model, but of the
example model.
The example model, on the other hand, is just an SBasePlugin, which is not an
\SBase{} and also not a children of the original model.
To ``inject'' the ListOfFoos in the right place, all children of the example
model in Java become direct children of the original model in XML.

All other objects that implement \SBase{} also implement \TreeNode, so we just add
all of their children to the list of elements to write.

\begin{figure}[htb]
  \begin{example}[numbers=left]
public ArrayList<Object> getListOfSBMLElementsToWrite(Object sbase) {

  if (logger.isDebugEnabled()) {
    logger.debug("getListOfSBMLElementsToWrite : " + sbase.getClass().getCanonicalName());
  }

  ArrayList<Object> listOfElementsToWrite = new ArrayList<Object>();

  if (sbase instanceof SBMLDocument) {
    // nothing to do
    // TODO : the 'required' attribute is written even if there is no plugin class for the SBMLDocument, 
    // so I am not totally sure how this is done.
  }
  else if (sbase instanceof Model) {
    ExampleModel modelGE = (ExampleModel) ((Model) sbase).getExtension(ExampleConstant.namespaceURI);

    Enumeration<TreeNode> children = modelGE.children();

    while (children.hasMoreElements()) {
      listOfElementsToWrite.add(children.nextElement());
    }
  }
  else if (sbase instanceof TreeNode) {
    Enumeration<TreeNode> children = ((TreeNode) sbase).children();

    while (children.hasMoreElements()) {
      listOfElementsToWrite.add(children.nextElement());
    }
  }

  if (listOfElementsToWrite.isEmpty()) {
    listOfElementsToWrite = null;
  } else if (logger.isDebugEnabled()) {
    logger.debug("getListOfSBMLElementsToWrite size = " + listOfElementsToWrite.size());
  }

  return listOfElementsToWrite;
}\end{example}
  \caption{Extension parser: \code{getListOfSBMLElementsToWrite()}.}
  \label{lst:ModelExtParserListSBMLToWrite}
\end{figure}

In some cases it may be necessary to modify the \code{writeElement()}
method.  For example, this can happen when the same Java class is mapped to
different XML tags, e.g., a default element and multiple additional tags.
If this would be represented not via an attribute, but by using different
tags, one could alter the name of the XML object in this method.

The actual writing of XML attributes must be implemented in each of the
classes in the \code{writeXMLAttributes()}.  An example is shown in 
\vref{lst:ModelExtCreateXMLAttributes} for the class \code{Foo}.

\begin{figure}[htb]
  \begin{example}[numbers=left]
public class Foo extends AbstractNamedSBase {
  ...

  public Map<String, String> writeXMLAttributes() {
    Map<String, String> attributes = super.writeXMLAttributes();
    if (isSetBar()) {
      attributes.remove("bar");
      attributes.put(Foo.shortLabel + ":bar", getBar());
    }

    // ...
    // further class attributes
  }
}\end{example}
  \caption{Method to write the XML attributes.}
  \label{lst:ModelExtCreateXMLAttributes}
\end{figure}


\subsubsection{Parsing}

The \code{processStartElement()} method is responsible for handling start
elements, such as \code{<listOfFoos>}, and creating the appropriate
objects.  The \code{contextObject} is the object representing the parent
node of the tag the parser just encountered.  First, you need to check for
every class that may be a parent node of the classes in your extension.  In
this case, those are objects of the classes \Model, \code{Foo} and
\code{ListOf}.  Note, that the \code{ExampleModel} has no corresponding XML
tag and the core model is already handled by the core parser.  This also
means that the context object of a ListOfFoos is not of the type
\code{ExampleModel}, but of type \Model.  But since the \code{ListOfFoos}
can only be added to an \code{ExampleModel}, the extension is retrieved or
created on the fly.

The \code{groupList} variable is used to keep track of where we are in
nested structures.  If the \code{listOfFoos} starting tag is encountered,
the corresponding enum value is assigned to that variable.  Due to Java's
type erasure, the context object inside a listOfFoos tag is of type
\code{ListOf<?>} and a correctly set \code{groupList} variable is the only
way of knowing where we are.  If we have checked that we are, in fact,
inside a \code{listOfFoos} node and encounter a \code{foo} tag, we create
\code{Foo} object and add it to the example model.  Technically, it is
added to the \code{ListOfFoos} of the example model, but since
\code{ExampleModel} provides convenience methods for managing its lists, it
is easier to call the \code{addFoo()} method on it.

\begin{figure}[htb]
  \begin{example}[numbers=left]
// Create the proper object and link it to his parent.
public Object processStartElement(String elementName, String prefix,
    boolean hasAttributes, boolean hasNamespaces, Object contextObject)
{

  if (contextObject instanceof Model) {
    Model model = (Model) contextObject;
    ExampleModel exModel = null;

    if (model.getExtension(ExampleConstant.namespaceURI) != null) {
      exModel = (ExampleModel) model.getExtension(ExampleConstant.namespaceURI);
    } else {
      exModel = new ExampleModel(model);
      model.addExtension(ExampleConstant.namespaceURI, exModel);
    }

    if (elementName.equals("listOfFoos")) {

      ListOf<Foos> listOfFoos = exModel.getListOfFoos();
      this.groupList = QualList.listOfFoos;
      return listOfFoos;
    }
  } else if (contextObject instanceof Foo) {
    Foo foo = (Foo) contextObject;

    // if Foo would have children, that would go here

  }
  else if (contextObject instanceof ListOf<?>)
  {
    ListOf<SBase> listOf = (ListOf<SBase>) contextObject;

    if (elementName.equals("foo") && this.groupList.equals(QualList.listOfFoos)) {
      Model model = (Model) listOf.getParentSBMLObject();
      ExampleModel exModel = (ExampleModel) model.getExtension(ExampleConstant.namespaceURI);

      Foo foo = new Foo();
      exModel.addFoo(foo);
      return foo;
    }
  }
  return contextObject;
}\end{example}
  \caption{Extension parser: \code{processStartElement()}.}
  \label{lst:ModelExtParserStartElement}
\end{figure}

The processEndElement() method is called whenever a closing tag is
encountered.  The groupList attribute needs to be updated to reflect the
step up in the tree of nested elements.  In this example, if the end of
\code{</listOfFoos>} is reached, we certainly are inside the model tags
again, which is denoted by \emph{none}.  Of course, more complicated
extensions with lots of nested lists need a more elaborate handling here,
but it should still be straight-forward.

\begin{figure}[htb]
  \begin{example}[numbers=left]
public boolean processEndElement(String elementName, String prefix,
  boolean isNested, Object contextObject) {

  if (elementName.equals("listOfFoos")
  {
    this.groupList = QualList.none;
  }

  return true;
}\end{example}
  \caption{Extension parser: \code{processEndElement()}.}
  \label{lst:ModelExtParserEndElement}
\end{figure}

Attributes of a tag are read into the corresponding object via the
\code{readAttributes()} method that must be implemented for each class.  An
example is shown in \vref{lst:ModelExtReadAttributes} for the class
\code{Foo}.

\begin{figure}[htb]
  \begin{example}[numbers=left]
@Override
public boolean readAttribute(String attributeName, String prefix, String value) {

  boolean isAttributeRead = super.readAttribute(attributeName, prefix, value);

  if (!isAttributeRead) {
    isAttributeRead = true;

    if (attributeName.equals(ExampleConstant.bar)) {
      setBar(StringTools.parseSBMLInt(value));
    } else {
      isAttributeRead = false;
    }
  }

  return isAttributeRead;
}\end{example}
  \caption{Method to read the XML attributes.}
  \label{lst:ModelExtReadAttributes}
\end{figure}


\section{Implementation checklist}

\Vref*{fig:checklist} presents a checklist summarizing the different
aspects of an extension that need to be implemented.

\newcommand{\fooname}{\code{\emph{\underline{\color{winered}Foo}}}}
\newcommand{\barname}{\code{\emph{\underline{\color{winered}Bar}}}}

\begin{figure}[htb]
  \begin{framed}
    \begin{itemize}[label=$\Box$,leftmargin=2em]

    \item Add the extension to an existing model (see
      \vref{lst:ModelExtClass}).

    \item Add the five necessary methods for each class attribute:

      \begin{itemize}[label=$\Box$]
        
      \item \code{get\barname()}
        
      \item \code{is\barname{}Mandatory()}
        
      \item \code{isSet\barname{}\fooname()} (only required if the attribute is an id)
        
      \item \code{set\barname(int value)}
        
      \item \code{unset\barname()}

      \end{itemize}

    \item Add the default constructors (see \vref{lst:ModelExtFooConstructors}).

    \item If the class has children, check if all list methods are implemented
      (see the program fragments in \ref{lst:ModelExtChildren},
      \ref{lst:ModelExtListOfFoosBasic}, \ref{lst:ModelExtAddRemoveFoos},
      \ref{lst:ModelExtChildren}):

      \begin{itemize}[label=$\Box$]
        
      \item \code{isSetListOf\fooname{}s()}
        
      \item \code{getListOf\fooname{}s()}
        
      \item \code{setListOf\fooname{}s(ListOf<\fooname{}> listOf\fooname{}s)}
        
      \item \code{add\fooname(\fooname{} foo)}
        
      \item \code{remove\fooname(\fooname{} foo)}
        
      \item \code{remove\fooname(int i)}
        
      \item \code{getAllowsChildren()}
        
      \item \code{getChildCount()}

      \end{itemize}

    \item All necessary create methods are implemented (see 
      \vref{lst:ModelExtCreateMethods}).

    \item Implement the \code{equals()} method (see 
      \vref{lst:ModelExtEquals}).

    \item Implement the \code{hashCode()} method (see 
      \vref{lst:ModelExtHashCode}).

    \item Implement the \code{clone()} method (see 
      \vref{lst:ModelExtClone} and \vref{lst:ModelExtCloneFoo}).

    \item Implement the \code{toString()} method.

    \item Implement the \code{writeXMLAttribute()} method (see 
      \vref{lst:ModelExtCreateXMLAttributes}).

    \item Implement the parser/writer method (see 
      \vref{lst:ModelExtParserClass},
      \vref{lst:ModelExtParserListSBMLToWrite},
      \vref{lst:ModelExtParserStartElement}, and
      \vref{lst:ModelExtParserEndElement}).
    \end{itemize}
  \end{framed}
  \caption{Implementation checklist for JSBML extension authors.}
  \label{fig:checklist}
\end{figure}
