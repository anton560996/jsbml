% -*- TeX-master: "User_guide"; fill-column: 75 -*-

For questions regarding SBML, please see the SBML FAQ at
\url{http://sbml.org/Documents/FAQ}.

\begin{description}

\item[\parbox{\textwidth}{Why does the class \LocalParameter not inherit
    from \Parameter?}]

  The reason is the \Boolean attribute \code{constant}, which is present
  in \index{constant} \index{parameter!\code{constant}}
  \Parameter and can be set to \code{false}. A parameter in the meaning of
  SBML is not always a constant, it might be some system variable \Variable
  and can therefore be the subject of \code{Rule}s, \index{rule} \Event{}s,
  \InitialAssignment{}s and so on, i.e., all instances of
  \code{Assignment}, whereas a \LocalParameter is defined as a constant
  quantity that never changes its value during the evaluation of a
  model\index{model}. It would therefore only be possible to let \Parameter
  inherit from \LocalParameter but this could lead to a semantic
  misinterpretation.

\item[\parbox{\textwidth}{Does JSBML depend on SWING or any particular
    graphical user interface implementation?}]

  Although all classes implement the \TreeNode interface (defined in
  the package \code{javax.swing.tree}), all classes in JSBML are entirely
  independent from any graphical user interface, such as the
  SWING\index{graphical user interface!swing@\code{swing}}
  implementation. When loading the \TreeNode interface, no other
  class from SWING will be initialized or loaded; hence JSBML can also be
  used on computers that do not provide any graphical system without the
  necessity of catching a \HeadlessException. The \TreeNode
  interface only defines methods and properties that all recursive tree data
  structures have to implement anyway. Letting JSBML classes extend this
  interface makes JSBML compatible with many other Java classes and methods
  that make use of the standard \TreeNode interface, hence ensuring a
  high compatibility with other Java libraries. Since the SWING package
  belongs to the standard Java distribution, the
  \TreeNode interface should always be localized by the Java Virtual
  Machine, independent from the specific hardware or
  system. Android\index{Android} systems might be an exceptional case, which
  do not provide any parts from the SWING package of Java. Therefore, the
  JSBML team is currently developing a specialized \code{android}
  compatibility module for JSBML. As discussed in \vref{sec:jsbml-modules},
  you can obtain this module by checking out the repository
  \url{https://jsbml.svn.sourceforge.net/svnroot/jsbml/modules/android} or by
  downloading this as a binary from the download page of JSBML.

\item[\parbox{\textwidth}{Does the usage of the the \code{java.beans}
    package for the \code{TreeNodeChangeListener} lead to an
    incompatibility with light-weight Java installations?}]

  With the \code{java.beans} package being part of the standard Java
  distribution, such an incompatibility will not occur. Extending existing
  standard Java classes leads to a higher compatibility with other
  libraries and should therefore be the preferred way to go in the
  development of JSBML.

\item[\parbox{\textwidth}{Does JSBML support SBML extension packages?}]

  In version 0.8, JSBML did not provide an abstract programming interface
  for extension packages.%
  \index{extension packages} Since then, the JSBML community has
  actively developed extension packages for the following SBML extensions:
  \code{fbc}, \code{groups}, \code{layout}, \code{multi},
  \code{qual}, and \code{spatial}. These packages can be used with the
  version 1.0 or later of JSBML.

\end{description}
