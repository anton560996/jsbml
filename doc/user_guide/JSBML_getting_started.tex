% -*- TeX-master: "User_guide"; fill-column: 75 -*-

\section{Obtaining and using JSBML}
\label{sec:obtaining-jsbml}

We provide four options for obtaining a copy of JSBML: (1) download the JAR
file distribution for JSBML complete with dependencies, that is, packaged
with third-party Java libraries needed by JSBML; (2) download the JAR file
distribution for JSBML \emph{excluding} dependencies; (3) download the
source code distribution; and (4) obtain the source code directly from the
project's Subversion repository. These four options are described below.


\subsection{The JSBML archive with dependencies}

The version of the archive that includes dependencies is a merged JAR file
that contains all of JSBML's required third-party libraries. You can
download it from the JSBML area on SourceForge~\cite{JSBMLdownload}. Once
you have installed the JAR file on your computer, it is sufficient to add
it to your Java build and/or class path in order to use JSBML.

\subsection{The JSBML archive without dependencies}

\begin{table}[b]
  \caption{List of other, third-party libraries needed by JSBML.}
  \label{tab:dependencies}
  \centering
  \rowcolors{2}{sbmlrowgray}{}
  \renewcommand{\arraystretch}{1.1}
  \begin{tabular}{>{\ttfamily}lp{2.26in}l}
    \toprule
    \textbf{\sffamily Library name} & \textbf{Purpose} & \textbf{Source URL} \\
    \midrule
    biojava-ontology-4.0.0.jar
    & biojava ontology-related classes~\citep{Holland2008}.
    \index{Ontology}
    & \href{http://biojava.org}{biojava.org} \\

    junit-4.8.jar
    & Unit-test support library; only needed if you \mbox{intend} to
    run the tests in the \code{tests} folder.
    & \href{http://www.junit.org}{www.junit.org} \\

    stax2-api-3.1.4.jar
    & Used for reading and writing XML.
    & \href{http://docs.codehaus.org/display/WSTX/StAX2}{docs.codehaus.org/display/WSTX/StAX2} \\

    woodstox-core-5.0.1.jar
    & Used for reading and writing XML.
    & \href{http://woodstox.codehaus.org}{woodstox.codehaus.org} \\
    
    staxmate-2.3.0.jar
    & Used for reading and writing XML. Provides a more user-friendly StAX
    interface.
    & \href{http://staxmate.codehaus.org}{staxmate.codehaus.org} \\

    xstream-1.3.1.jar
    & Used for reading and writing XML, specifically parsing results from
    the SBML validator.
    & \href{http://xstream.codehaus.org}{xstream.codehaus.org} \\

    jigsaw-dateParser.jar
    & Portion of the \emph{Jigsaw} library (version from
    Dec. 2010), containing classes for date manipulation. 
    & \href{http://jigsaw.w3.org}{jigsaw.w3.org} \\

    log4j-1.2-api-2.3.jar 
    & Libraries for logging errors and other diagnostics.
    & \href{http://logging.apache.org/log4j}{logging.apache.org/log4j} and \href{http://www.slf4j.org}{slf4j.org} \\

    log4j-api-2.3.jar \\ log4j-core-2.3.jar \\ log4j-slf4j-impl-2.3.jar \\ slf4j-api-1.7.21.jar \\

    \bottomrule
  \end{tabular}
\end{table}

The version of the JSBML archive that excludes dependencies is a JAR file
that contains only JSBML classes. You can download it from the JSBML area
on SourceForge~\cite{JSBMLdownload}. Since it does not include the
third-party libraries needed by JSBML to operate, you will need to obtain
and download those libraries separately. \tab{tab:dependencies} lists
what they are. Once you have installed the JSBML JAR file \emph{and} these
third-party libraries on your computer, you will need to add them
\emph{all} to your Java build and/or class path in order to use JSBML.


\subsection{Maven dependencies}

JSBML can also be obtained through Apache Maven~\citep{ApacheMaven}.
If you are already using Maven in your project, you can add JSBML
as a dependency, just add the following lines into your pom.xml:

\begin{example}[language=XML, title={Maven instructions to add to your pom.xml.}]
  <repositories>
    <repository>
      <id>ebi-repo</id>
      <name>The EBI repository</name>
      <url>http://www.ebi.ac.uk/~maven/m2repo</url>
      <releases><enabled>true</enabled></releases>
    </repository>
  </repositories>

  <dependencies>
    <dependency>
      <groupId>org.sbml.jsbml</groupId>
      <artifactId>jsbml</artifactId>
      <version>|\jsbmlversion|</version>
    </dependency>
  </dependencies> 
\end{example}

The jsbml artifact will include jsbml-core plus all L3 packages. Like this
there is not need to list all the packages by hand and when
a new one is developed, you will get it without having to make too much change to your
pom files. 

If you want to pick which package to include, you can list them one by one, there are
further instructions at \url{http://sbml.org/Software/JSBML/docs/Maven_Configuration}.


\subsection{The JSBML source archive}
\label{sec:jsbml-source-archive}

The source distribution for JSBML is similar to the JAR distribution that
excludes third-party dependency libraries, except that the JSBML files are
not compiled into class files; you must compile them yourself. As with the
other options described above, the source distribution is available from
the JSBML area on SourceForge~\cite{JSBMLdownload}, as an archive file in
ZIP format.

Download whichever format is more convenient for you and unpack the archive
on your computer somewhere.  The act of unpacking the archive will create a
folder on your computer named after the distribution version; for
example, this may be ``\code{jsbml-}\jsbmlversion''.  Next, you will need to compile
the Java source code.  JSBML comes with a \emph{build file} (i.e., scripted
instructions in a specialized format) for Apache Ant~\citep{ApacheAnt};
you can use other approaches for compiling the JSBML classes and
performing other tasks, but Ant provides an especially convenient approach.
For the rest of the instructions below, we use Ant.  Here is an example of
how to compile the JSBML class files after you have unpacked the source
code archive:

\begin{example}[style=bash, title={Compiling JSBML with Ant; this example
    uses Bash shell syntax.}] 
cd jsbml-|\jsbmlversion|
ant compile
\end{example}

Next, if you wish to run the self-tests included with JSBML, you can do so by
running the following command:

\begin{example}[style=bash, title={Running the unit tests provided with JSBML.}]
ant test
\end{example}

Finally, if you want to produce a JAR file containing all the JSBML
compiled class files, run the following command:

\begin{example}[style=bash, title={Creating a JAR file.}]
ant jar
\end{example}



\subsection{The JSBML source code repository}
\label{sec:SourceDistribution}

The fourth approach to obtaining a copy of JSBML is to retrieve it directly
from the Subversion repository~\cite{JSBMLSVN}.  Here is an example of how
to retrieve the latest version of the JSBML sources:

\newcommand{\dirname}{\code{\emph{\color{winered}jsbml}}\xspace}

\begin{example}[style=bash, title={Downloading the latest JSBML 
    sources from the JSBML project's Subversion repository.}]
svn co https://svn.code.sf.net/p/jsbml/code/trunk |\dirname|
cd |\dirname|
\end{example}

(The name you give to the copy on your computer is up to you.  We used
``\dirname'' in this example, but you could name the folder something else
if you wish.)  Once you have retrieved the folder from the Subversion
repository, you can compile the source files and create a JAR file.  Please
refer to the instructions in \sec{sec:jsbml-source-archive}.

The Subversion repository contains copies of all the third-party libraries
listed in \tab{tab:dependencies} and needed by JSBML.  They are
located in the folder ``\dirname''\code{/core/lib}.


\subsection{Setting up Eclipse}
\label{sec:SettingUpEclipse}

To set up Eclipse to work with JSBML, first add the \texttt{src},
\texttt{test} and \texttt{resources} folder of the JSBML distribution to your
Eclipse build path, and add all of the \texttt{.jar} files found in the
\texttt{lib} folder.

Next, you need to do an extra step to configure the annotation processor,
because the different parsers in JSBML are registered automatically using
Java annotations.  To configure the annotation processor in Eclipse, follow
the instructions found on the web page
\url{https://github.com/niko-rodrigue/spi/blob/wiki/EclipseSettings.md}.
 The jar file of the annotation processor is located in the jsbml source tree at
 ``\dirname''\code{/core/lib/spi-full-0.2.4.jar}. If you checkout 
the full trunk (and not just the core), you can find a folder dev, which 
contains a README.txt file that has also these instructions and further 
important information.

Finally, you can run the \code{ParserManager} class to check that the list of
parsers are not empty and that they contain the parsers you need.


\subsection{Optional extensions, modules and examples available for JSBML}
\label{sec:dependencies}

JSBML provides a number of additional extensions, modules and example
programs that you may find useful in your work.  The \emph{extensions} are
optional add-ons that implement support for SBML Level~3 Packages; these
packages extend SBML syntax to support, for example, storing the layout of
a model's graphical diagram directly in the SBML file.  The JSBML
\emph{modules} provide additional features and interfaces, for example, to
allow CellDesigner~\cite{Funahashi2003} plugins to use JSBML.  Finally, the
JSBML \emph{examples} are full-fledged applications that demonstrate the
use of JSBML in actual running software.  Each of these optional components
of JSBML is available from the project's code repository (and in some
cases, from the download area on SourceForge~\cite{JSBMLdownload}).  In the
subsections below, we explain how to obtain copies of them from the
repository.


\subsubsection{JSBML Extensions}

The JSBML repository's \code{extensions} folder contains a separate
subfolder for each currently-implemented JSBML extension. You can either
retrieve a copy of each extension separately, or obtain the complete
\code{extensions} portion of the repository.  Here we explain the latter.

First, find a suitable location on your computer where you would like to
place the JSBML extensions folder.  (We suggest placing it side-by-side at
the same level as your JSBML core folder, e.g., next to the folder
``\dirname'' discussed above.)  Then, perform the following step:

\begin{example}[style=bash, title={Downloading the latest JSBML extensions
    source folder from the project's Subversion repository.}] 
svn co https://svn.code.sf.net/p/jsbml/code/trunk/extensions jsbml-extensions
\end{example}

Each of the extensions has its own Ant build script, located in a file
named (as per Ant conventions) ``\code{build.xml}'' within the extension's
subfolder.  To build, for example, the \code{layout} extension, you could
do the following:

\begin{example}[style=bash, title={Compiling the ``\code{layout}'' extension.}]
cd extensions/layout
ant compile
\end{example}


\subsubsection{JSBML Modules}
\label{sec:jsbml-modules}

JSBML currently provides six additional modules.  Each provides features
for task-specific purposes. Binary versions of the modules can be found at
the download site of JSBML~\cite{JSBMLdownload}; here we explain how you
can obtain the most recent versions of the modules directly from the source
code repository.  (Note: at the time of this writing, only the
\code{tidy}, \code{CellDesigner} and the \code{libSBMLio} module have been extensively
tested.)

First, find a suitable location on your computer where you would like to
place the JSBML modules folder.  We suggest creating a folder named
``\code{modules}'' placed side-by-side at the same level as your JSBML core
folder, e.g., next to the folder ``\dirname'' discussed above.

\begin{example}[style=bash, title={Creating a folder for the modules.}]
mkdir modules
cd modules
\end{example}

\newcommand{\modulename}{\code{\emph{\color{winered}modulename}}\xspace}

Next, perform the following operation, once for each of the modules you
would like to obtain, where the variable \modulename is one of the names
listed in the first column of \tab{tab:jsbml-modules}:

\begin{example}[style=bash, title={Obtaining a JSBML module.}]
svn co https://svn.code.sf.net/p/jsbml/code/trunk/modules/|\modulename| |\modulename|
\end{example}

(In other words, if you would like to obtain both the Android and libSBMLio
modules, execute the command above twice, once with \code{android} in place
of \modulename and a second time with \code{libSBMLio} in place of
\modulename.)  Once they're downloaded, please check inside each module
directory for information about how to use them.

\begin{table}[thb]
  \caption{JSBML modules available today.}
  \label{tab:jsbml-modules}
  \centering
  \rowcolors{2}{sbmlrowgray}{}
  \begin{tabular}{>{\ttfamily}lp{5.25in}}
    \toprule
    \textbf{\sffamily Module name} & \textbf{Purpose} \\
    \midrule
    android
    & Support for writing JSBML-based programs for Android OS.
    \\
    celldesigner
    & A bridge module that supports writing JSBML-based
    plugins for CellDesigner~\cite{Funahashi2003}
    \\
    compare
    & Facilities for doing comparisons between libSBML and JSBML
    \\
    libSBMLcompat
    & A module that allows easier switching between libSBML and JSBML by
    providing wrapper classes replicating much of libSBML's API in JSBML (in development)
    \\
    libSBMLio
    & A libSBML communications layer.
    \\
    tidy
    & A warper around the SBMLWriter class that use the jtidy library~\cite{jtidy} (a Java port of HTML Tidy) to
    format properly the resulting XML.
    \\
    \bottomrule
  \end{tabular}
\end{table}

You can find more information and explanation about JSBML's modules in \sec{sec:jsbml-modules-details}.


\subsubsection{JSBML Examples}
\label{sec:jsbml-repo-examples}

The JSBML repository's \code{examples} folder contains a separate
subfolder for each sample application.  To obtain them, first, find a
suitable location on your computer where you would like to place the JSBML
examples folder.  We suggest creating a folder named ``\code{examples}''
placed side-by-side at the same level as your JSBML core folder.

\begin{example}[style=bash, title={Creating a folder for the examples.}]
mkdir examples
cd examples
\end{example}

Next, retrieve the examples you would like to obtain.  At the time of this
writing, there is only one example available:

\begin{example}[style=bash, title={Retrieving the \emph{SBML Bar Graph}
    example application.}] 
svn co https://svn.code.sf.net/p/jsbml/code/trunk/examples/sbmlbargraph sbmlbargraph
\end{example}

Finally, please read the ``\code{README.txt}'' file in the freshly-obtained
\code{sbmlbargraph} folder to learn more about how to get started with the
example application.


