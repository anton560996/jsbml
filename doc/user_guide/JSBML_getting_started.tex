% -*- TeX-master: "User_guide"; fill-column: 75 -*-

\section{Obtaining and using JSBML}
\label{sec:obtaining-jsbml}

We provide four options for obtaining a copy of JSBML: (1) download the JAR
file distribution for JSBML complete with dependencies, that is, packaged
with third-party Java libraries needed by JSBML; (2) download the JAR file
distribution for JSBML \emph{excluding} dependencies; (3) download the
source code distribution; and (4) obtain the source code directly from the
project's GitHub~\citep{JSBMLGIT} repository. These four options are described below.


\subsection{The JSBML archive with dependencies}

The version of the archive that \emph{includes} dependencies is a merged JAR file
that contains all of JSBML's required third-party libraries. You can download
it from the JSBML download directories on both GitHub and SourceForge~(\cite{JSBMLdownloadSF,
  JSBMLdownloadGitHub}). Once you have installed the
JAR file on your computer, it is sufficient to add it to your project's Java build
and/or class path in order to use JSBML.

\subsection{The JSBML archive without dependencies}

\begin{table}[b]
\vspace*{-1em}
  \caption{List of other, third-party libraries needed by JSBML.}
  \label{tab:dependencies}
  \centering
  \rowcolors{2}{sbmlrowgray}{}
  \setlength{\extrarowheight}{1pt}
  \renewcommand{\arraystretch}{1.1}
  \begin{tabular}{>{\ttfamily}lm{2.25in}l}
    \toprule
    \textbf{\sffamily{Library name}} & \textbf{Purpose} & \textbf{Source URL} \\
    \midrule
    biojava-ontology-4.0.0.jar
    & biojava ontology-related classes~\citep{Holland2008}.
    \index{Ontology}
    & \href{http://biojava.org}{biojava.org} \\

    junit-4.8.jar
    & Unit-test support library; only needed if you \mbox{intend} to
    run the tests in the \code{tests} folder.
    & \href{http://www.junit.org}{www.junit.org} \\

    stax2-api-3.1.4.jar
    & Used for reading and writing XML.
    & \href{http://docs.codehaus.org/display/WSTX/StAX2}{docs.codehaus.org/display/WSTX/StAX2} \\

    woodstox-core-5.0.1.jar
    & Used for reading and writing XML.
    & \href{http://woodstox.codehaus.org}{woodstox.codehaus.org} \\
    
    staxmate-2.3.0.jar
    & Used for reading and writing XML. Provides a more user-friendly StAX
    interface.
    & \href{http://staxmate.codehaus.org}{staxmate.codehaus.org} \\

    xstream-1.3.1.jar
    & Used for reading and writing XML, specifically parsing results from
    the SBML validator.
    & \href{http://xstream.codehaus.org}{xstream.codehaus.org} \\

    jigsaw-dateParser.jar
    & Portion of the \emph{Jigsaw} library (version from
    Dec. 2010), containing classes for date manipulation. 
    & \href{http://jigsaw.w3.org}{jigsaw.w3.org} \\

    \parbox[c]{1.4in}{log4j-1.2-api-2.3.jar\par
      log4j-api-2.3.jar\par
      log4j-core-2.3.jar\par
      log4j-slf4j-impl-2.3.jar\strut}
    & Libraries for logging diagnostics.
    & \href{http://logging.apache.org/log4j}{logging.apache.org/log4j} \\

    slf4j-api-1.7.21.jar
    & Logging interface library.
    & \href{http://www.slf4j.org}{slf4j.org} \\

    \bottomrule
  \end{tabular}
\end{table}

The version of the JSBML archive that excludes dependencies is a JAR file
that contains only JSBML classes. You can download it from the JSBML download areas
on SourceForge and GitHub (\cite{JSBMLdownloadSF,JSBMLdownloadGitHub}). Since it does not include the
third-party libraries needed by JSBML to operate, you will need to obtain
and download those libraries separately. \tab{tab:dependencies} lists
what they are. Once you have installed the JSBML JAR file \emph{and} these
third-party libraries on your computer, you will need to add them
\emph{all} to your project's Java build and/or class path in order to use JSBML.


\subsection{Maven dependencies}

JSBML can also be obtained through Apache Maven~\citep{ApacheMaven}.
If you are already using Maven in your project, you can add JSBML
as a dependency by adding these lines into your project's \texttt{pom.xml} file:

\begin{example}[style=XML,
keywords={repositories,repository,id,name,url,releases,enabled,dependencies,
    dependency,groupId,artifactId,version},
  title={Maven instructions to add to your pom.xml.}]
<dependencies>
  <dependency>
    <groupId>org.sbml.jsbml</groupId>
    <artifactId>jsbml</artifactId>
    <version>|\jsbmlversion|</version>
  </dependency>
</dependencies> 
\end{example}

The \texttt{jsbml} artifact will include \texttt{jsbml-core} plus the JSBML
extensions that support all available SBML Level~3 packages.  With this
approach, there is no need to list all the JSBML extensions by hand, and when
a new one is developed, you will get it without having to make too many
changes to your \texttt{pom.xml} files.

If you want to select the JSBML extensions that get included in your
project, you can opt to list them one by one, although this is not recommended practice.  Instructions for doing this can be
found at \url{http://sbml.org/Software/JSBML/docs/Maven_Configuration}.


\subsection{The JSBML source archive}
\label{sec:jsbml-source-archive}

The source distribution for JSBML is similar to the JAR distribution that
excludes third-party dependency libraries, except that the JSBML files are
not compiled into class files; you must compile them yourself. As with the
other options described above, the source distribution is available from
the JSBML download areas on SourceForge and
GitHub~(\cite{JSBMLdownloadSF, JSBMLdownloadGitHub}), as an archive file in
either ZIP or gzip'ed TAR format.

You may download the archive in whichever format is more convenient for
you, and unpack it
on your computer somewhere.  The act of unpacking the archive will create a
folder on your computer named after the distribution version; for
example, this may be ``\code{jsbml-}\jsbmlversion''.  Next, compile
the Java source code.  JSBML comes with a \emph{build file} for Apache Ant~\citep{ApacheAnt};
you can use other approaches for compiling the JSBML classes and
performing other tasks, but Ant provides an especially convenient approach.
For the rest of the instructions below, we use Ant.  Here is an example of
how to compile the JSBML class files after you have unpacked the source
code archive:

\begin{example}[style=bash, title={Compiling JSBML with Ant; this example
    uses Bash shell syntax.}] 
cd jsbml-|\jsbmlversion|
ant compile
\end{example}

Next, if you wish to run the self-tests included with JSBML, you can do so by
running the following command:

\begin{example}[style=bash, title={Running the unit tests provided with JSBML.}]
ant test
\end{example}

Finally, if you want to produce a JAR file containing all the JSBML
compiled class files, run the following command:

\begin{example}[style=bash, title={Creating a JAR file.}]
ant jar
\end{example}



\subsection{The JSBML source code repository}
\label{sec:SourceDistribution}

The fourth approach to obtaining a copy of JSBML is to retrieve it directly
from the project GitHub repository~\citep{JSBMLGIT}.  Here is an example of how
to retrieve the latest version of the JSBML sources:

\newcommand{\dirname}{\code{\emph{\bfseries\color{winered}jsbml}}\xspace}

\begin{example}[style=bash, title={Downloading the latest JSBML 
    sources from the JSBML project's GitHub repository.}]
git clone --recursive git@github.com:sbmlteam/jsbml.git |\dirname|
cd |\dirname|
\end{example}

(The name you give to the copy on your computer is up to you.  We used
``\dirname'' in this example, but you could name the folder something else
if you wish.)  Once you have retrieved the folder from the
repository, you can compile the source files and create a JAR file.  Please
refer to the instructions in \sec{sec:jsbml-source-archive}.

The JSBML git repository contains copies of all the third-party libraries
listed in \tab{tab:dependencies} and needed by JSBML.  They are
located in the folder ``\dirname''\code{/core/lib}.


\subsection{Setting up Eclipse}
\label{sec:SettingUpEclipse}

To set up Eclipse to work with JSBML, first add the \texttt{core/src},
\texttt{core/test} and \texttt{core/resources} folder of the JSBML distribution to your
Eclipse build path, and add all of the \texttt{.jar} files found in the
\texttt{core/lib} folder.

Next, you need to do an extra step to configure the annotation processor,
because the different parsers in JSBML are registered automatically using
Java annotations.  To configure the annotation processor in Eclipse, follow
the instructions given on the web page
\url{https://github.com/niko-rodrigue/spi/blob/wiki/EclipseSettings.md}.
 The JAR file of the annotation processor is located in the JSBML source tree at
 ``\dirname''\code{/core/lib/spi-full-0.2.4.jar}. If you cloned
the full JSBML source tree, you can find in it a folder named \texttt{dev}, which 
contains a \texttt{README.txt} file that has also these instructions and other
important information.
Finally, you can run the Eclipse \code{ParserManager} class to check that the list of
parsers is not empty and that it includes the parsers you need.


\subsection{Optional extensions, modules and examples available for JSBML}
\label{sec:dependencies}

JSBML provides a number of additional extensions, modules and example
programs that you may find useful in your work.  The JSBML \emph{extensions} are
optional add-ons that implement support for SBML Level~3 Packages; these
packages extend SBML syntax to support, for example, storing the layout of
a model's graphical diagram directly in the SBML file.  The JSBML
\emph{modules} provide additional features and interfaces, for example, to
allow CellDesigner~\cite{Funahashi2003} plugins to use JSBML.  Finally, the
JSBML \emph{examples} are full-fledged applications that demonstrate the
use of JSBML in actual running software.  Each of these optional components
of JSBML is available from the project's code repository (and in some
cases, from the download areas on SourceForge and
GitHub~\cite{JSBMLdownloadSF, JSBMLdownloadGitHub}).


\subsubsection{JSBML Extensions}

The \code{extensions} folder in the JSBML source tree contains a separate
subfolder for each currently implemented JSBML extension.
Each of these has its own Ant build script, located in a file
named ``\code{build.xml}'' within the extension's
subfolder.  To build, for example, the \code{layout} extension, you could
do the following:

\begin{example}[style=bash, title={Compiling the JSBML ``\code{layout}'' extension.}]
cd extensions/layout
ant compile
\end{example}

\vspace*{-2ex}

\subsubsection{JSBML Modules}
\label{sec:jsbml-modules}

The currently available modules are summarized in
\tab{tab:jsbml-modules}.  Binary versions of the modules can be
found at the JSBML download sites~(\cite{JSBMLdownloadSF,
  JSBMLdownloadGitHub}); you can also build them from the JSBML
source tree.  Within the \texttt{modules} folder, you will find a
separate subdirectory for each module.  Most have their own Ant build
scripts, located in a file named ``\code{build.xml}''.  You can
build a module by performing steps such as in the following
example:

\begin{example}[style=bash, title={Compiling the JSBML ``\code{layout}'' extension.}]
cd modules/tidy
ant jar
\end{example}

\begin{table}[thb]
  \caption{JSBML modules available today.}
  \label{tab:jsbml-modules}
  \centering
  \rowcolors{2}{sbmlrowgray}{}
  \begin{tabular}{>{\ttfamily}lp{5.25in}}
    \toprule
    \textbf{\sffamily Module name} & \textbf{Purpose} \\
    \midrule
    android
    & Support for writing JSBML-based programs for Android OS.
    \\
    celldesigner
    & A bridge module that supports writing JSBML-based
    plugins for CellDesigner~\cite{Funahashi2003}
    \\
    compare
    & Facilities for doing comparisons between libSBML and JSBML
    \\
    libSBMLcompat
    & A module that allows easier switching between libSBML and JSBML by
    providing wrapper classes replicating much of libSBML's API in JSBML (in development)
    \\
    libSBMLio
    & A libSBML communications layer.
    \\
    tidy
    & A warper around the SBMLWriter class that use the jtidy library~\cite{jtidy} to
    format properly the resulting XML.
    \\
    \bottomrule
  \end{tabular}
\end{table}

Note: at the time of this writing, only the \code{tidy},
\code{CellDesigner} and the \code{libSBMLio} module have been
tested extensively.  You can find more information and explanation
about the JSBML modules in \sec{sec:jsbml-modules-details}.


\subsubsection{JSBML Examples}
\label{sec:jsbml-repo-examples}

The \code{examples} folder contains a separate subfolder for each sample
application.  At the time of this writing, there is only one example
available.  Similar to the extensions and modules, you can build the sample
application from the source code.  Please refer to the
``\code{README.txt}'' file in the \code{examples/sbmlbargraph} folder to
learn more.
