% -*- TeX-master: "User_guide"; fill-column: 75 -*-

\section{Creating the object hierarchy}
\label{sec:creating-obj-hierarchy}

A JSBML extension may need to do different things depending on the details
of the \SBMLthree package that it implements.  In this section, we discuss
various common actions and how they can be written in JSBML.


\subsection{Introducing new components and extending others}
\label{subsec:addingClasses}

Most \SBMLthree packages extend existing SBML core components or define
entirely new components.  A common need for packages is to extend the SBML
\Model object, so we begin by explaining how this can be achieved.

\begin{figure}[t]
  \begin{example}[numbers=left]
public class ExampleModel extends AbstractSBasePlugin {

  // Basic constructor.
  public ExampleModel(Model model) {
    super(model);
  }

  // Returns the model.
  public Model getModel() {
    return (Model) getExtendedSBase();
  }
}\end{example}
  \caption{How to extending \AbstractSBasePlugin to create an extended
    \,\Model.}
  \label{lst:ModelExtClass}
\end{figure}

\fig{lst:ModelExtClass} shows the beginnings of the definition for
a class named \code{ExampleModel} that extends the plain SBML
\Model. Technically, an extension really only needs to implement the
\SBasePlugin interface, but because the abstract class \AbstractSBasePlugin
implements important and useful methods, it is generally preferable to
extend that one instead.  In this example, our constructor for
\code{ExampleModel} accepts an object that is a \Model, because that is
what we want to extend.  The call to the super constructor will save the
given model as the \SBase object that is being extended, and it will store
it in an attribute named \code{extendedSBase}. Our example
\code{ExampleModel} class also adds a method, \code{getModel()}, to
retrieve the extended model object.

In most cases, extensions will also introduce new components that have no
counterpart in the SBML core.  We illustrate this here by creating a
component called \code{Foo} with three attributes: \code{id}, \code{name},
and an integer-valued attribute, \code{bar}.  We assume that in the
(hypothetical) package specification for \emph{Example}, \code{Foo} is
derived from \SBase; let us also assume that \emph{Example} provides the
ability to attach a list of \code{Foo} objects to an extended version of
\Model.  We show in \sec{sec:listofs} how to create the list
structure; here, we focus on the definition of \code{Foo}.  We define the
class \code{Foo} by extending \AbstractSBasePlugin, and add methods for
working with the attributes.  In \fig{lst:DefaultModelExtFooBar},
we list the code so far, focusing on just one of the attributes,
\emph{bar}.

\begin{figure}[hbt]
  \begin{example}[numbers=left]
// Use Integer, so we can denote unset values as null public Integer bar;

public int getBar() {
  if (isSetBar()) {
    return bar.intValue();
  }
  // This is necessary because we cannot return null here.
  throw new PropertyUndefinedError(ExampleConstant.bar, this);
}

public boolean isSetBar() {
  return this.bar != null;
}

public void setBar(int value) {
  Integer oldBar = this.bar;
  this.bar = bar;
  firePropertyChange(ExampleConstant.bar, oldBar, this.bar);
}

public boolean unsetBar() {
  if (isSetBar()) {
    Integer oldBar = this.bar;
    this.bar = null;
    firePropertyChange(ExampleConstant.bar, oldBar, this.bar);
    return true;
  }
  return false;
}\end{example}
  \caption{Implementation of the five necessary methods that should be
    created for every attribute on class \code{Foo}.  Note: if attribute
    \emph{bar} had been a boolean-valued attribute, we would also provide
    the method \code{isBar()}, whose implementation would delegate to
    \code{getBar()}.}
  \label{lst:DefaultModelExtFooBar}
\end{figure}

A few points about the code of \fig{lst:DefaultModelExtFooBar} are worth
mentioning.  The identifiers on SBML components are often required to be
unique; for many components, the scope of uniqueness is the entire set of
main SBML components (e.g., \Species, \Compartment{}s, etc.), but some have
uniqueness requirements that are limited to some subset of entities (e.g.,
unit identifiers).  For the purposes of this example, we assume that the
identifiers of \emph{Foo} objects in a model must be unique across all
identifiers in the model.  All entities that have such uniqueness constraints
should implement the JSBML interface \UniqueNamedSBase; in our example, this
is taken care of by the abstract superclasses, so nothing needs to be done
explicitly here.

The code in \fig{lst:DefaultModelExtFooBar} illustrates
another point, the need call to \code{firePropertyChange()} in set and
unset methods.  This is needed in order to ensure that all listeners are
notified about changes to the objects.  Finally, note that in cases that
the return type is a Java base type, such as \code{int} or \code{boolean},
but the corresponding internal element (e.g., \code{Integer} or
\code{Boolean}) is set to \code{null}, the program must throw a
\code{PropertyUndefineError} in the get method to prevent incorrect
results (see line~8).

The last basic matter that needs to be addressed is the definition of
appropriate class constructors for our class \code{Foo}.  The minimum we
need to define is a constructor that takes no arguments.  Even though some
or all of the attributes of a class may be mandatory, default constructors
that take no arguments still need to be defined in JSBML.  This is due to
the internal working of parsers that read SBML files and create the data
structure in memory.  All attributes can be set after the object has been
created.

\begin{figure}[b]
  \begin{example}[numbers=left]
public Foo() {
  super();
  initDefaults();
}

public Foo(String id) {
  super(id);
  initDefaults();
}

public Foo(int level, int version){
  this(null, null, level, version);
}

public Foo(String id, int level, int version) {
  this(id, null, level, version);
}

public Foo(String id, String name, int level, int version) throws LevelVersionError {
  super(id, name, level, version);
  if (getLevelAndVersion().compareTo(Integer.valueOf(3), Integer.valueOf(1)) < 0) {
    throw new LevelVersionError(getElementName(), level, version);
  }
  initDefaults();
}

/** Clone constructor */
public Foo(Foo foo) {
  super(foo);
  bar = foo.bar;
}

public void initDefaults() {
  addNamespace(ExampleConstant.namespaceURI);
  bar = null;
}\end{example}
  \caption{Constructors for class \code{Foo}.  Note the code testing for
    the SBML Level and Version, on lines~21--23; since this extension
    implements a hypothetical package for \SBMLthree, the code here rejects
    anything before Level~3 Version~1 by throwing the JSBML exception
    \LevelVersionError.}
  \label{lst:ModelExtFooConstructors}
\end{figure}

Beyond this, the precise combination of constructor arguments defined for a
class is a design issue that must be decided for each class individually.
Attempting to define a separate constructor for every possible combination
of arguments can lead to a combinatorial explosion, resulting in complex
class definitions, confusing code, and excessive maintenance costs, so it
is better to decide which combinations of arguments are the most common and
focus on them.  In \fig{lst:ModelExtFooConstructors}, we show a
recommended selection of constructors.  They include a constructor that
takes an identifier only, another that takes SBML Level and Version values
only, and another that takes all arguments together.  If you delegate the
constructor call to the super class, you have to take care of the
initialization of your custom attributes yourself (by calling a method like
\code{initDefaults()}).  If you delegate to another constructor in your
class, you only have to do that at the last one in the delegation chain.


\subsection{\codeNC{ListOf}s}
\label{sec:listofs}

Our hypothetical \emph{Example} extension adds no new attributes to the
extended \Model itself, but it does introduce the ability to have a list of
\code{Foo} objects as a child of \Model.  In JSBML, this will be
implemented using Java generics and the class \code{ListOf}, such that the
type of the list will be \code{ListOf<Foo>}.  (Unlike in libSBML, there
will not be an actual separate \code{ListOfFoo} class.)  In
\fig{lst:ModelExtListOfFoosBasic}, we show the basic implementation
of the methods that would be added to \Model to handle
\code{ListOf<Foo>}: \code{isSetListOfFoos()}, \code{getListOfFoos()},
\code{setListOfFoos(ListOf<Foo>)}, and \code{unsetListOfFoos()}.

\begin{figure}[b]
  \begin{example}[numbers=left]
public boolean isSetListOfFoos() {
  return (listOfFoos != null) && !listOfFoos.isEmpty();
}

public ListOf<Foo> getListOfFoos() {
  if (!isSetListOfFoos()) {
    Model m = getModel();
    listOfFoos = new ListOf<Foo>(m.getLevel(), m.getVersion());
    listOfFoos.addNamespace(ExampleConstants.namespaceURI);
    listOfFoos.setSBaseListType(ListOf.Type.other);
    m.registerChild(listOfFoos);
  }
  return this.listOfFoos;
}

public void setListOfFoos(ListOf<Foo> listOfFoos) {
  unsetListOfFoos();
  this.listOfFoos = listOfFoos;
  getModel().registerChild(this.listOfFoos);
}

public boolean unsetListOfFoos() {
  if (isSetListOfFoos()) {
    ListOf<Foos> oldFoos = this.listOfFoos;
    this.listOfFoos = null;
    oldFoos.fireNodeRemovedEvent();
    return true;
  }
  return false;
}\end{example}
  \caption{Implementation of the methods \code{isSetListOfFoos()},
    \code{getListOfFoos()}, and \code{setListOfFoos()}.}
  \label{lst:ModelExtListOfFoosBasic}
\end{figure}

Typically, when adding and removing Foo objects to the \Model, direct
access to the actual \code{ListOf} object is not necessary.  To add and
remove \code{Foo} objects from a given SBML model, it is more convenient to
have methods to add and remove on \code{Foo} object at a time.  We show
such methods in \fig{lst:ModelExtAddRemoveFoos}.  The methods also
do some additional consistency checking as part of their work.

\begin{figure}[t]
  \begin{example}[numbers=left]
public boolean addFoo(Foo foo) {
    return getListOfFoos().add(foo);
}

public boolean removeFoo(Foo foo) {
  return isSetListOfFoos() ? getListOfFoos().remove(foo) : false;
}

public boolean removeFoo(int i) {
  if (!isSetListOfFoos()) {
    throw new IndexOutOfBoundsException(Integer.toString(i));
  }
  return listOfFoos.remove(i);
}

// If the object class has an id, one should also add the following:
public boolean removeFoo(String id) {
  return getListOfFoos().removeFirst(new NameFilter(id));
}\end{example}
  \caption{Implementation of \code{ListOf} methods \code{addFoo(Foo foo)},
    \code{removeFoo(Foo foo)}, \code{removeFoo(int i)}.}
  \label{lst:ModelExtAddRemoveFoos}
\end{figure}

\begin{figure}[hb]
  \begin{example}[numbers=left]
public boolean getAllowsChildren() {
  return true;
}

public int getChildCount() {
  int count = 0;
  if (isSetListOfFoos()) {
    count++;
  }
  return count;  // Same for each additional ListOf* or other subelement in this package.
}

public SBase getChildAt(int childIndex) {
  if (childIndex < 0) {
    throw new IndexOutOfBoundsException(childIndex + " < 0");
  }

  // Important: there must be an index shift according to the number of child elements in the superclass.

  int pos = 0;
  if (isSetListOfFoos()) {
    if (pos == childIndex) {
      return getListOfFoos();
    }
    pos++;
  }
  // Same for each additional ListOf* or other subelements in this package.
  throw new IndexOutOfBoundsException(MessageFormat.format(
    "Index {0,number,integer} >= {1,number,integer}", childIndex, ((int) Math.min(pos, 0))));
}\end{example}
  \caption{Methods which need to be implemented to make the children
    available in the extended model.}
  \label{lst:ModelExtChildren}
\end{figure}

To let a \code{ListOfFoo} appear as a child of the standard \Model, some
important methods for \TreeNode need to be implemented (see
\fig{lst:ModelExtChildren}).  Method \code{getAllowsChildren()} should
return \code{true} in this case, since this extension allows children.  The
child count and the indices of the children is a bit more complicated,
because they vary with the number of non-empty \code{ListOf}s.  So, for
every non-empty \code{ListOf} child of our model extension, we increase the
counter by one.  (Note also that if callers access list entries by index
number, they will need to take into account the possibility that a given
object's index may shift. )


\subsection{Methods for creating new objects}

Since a newly created instance of type \code{Foo} is not part of the model
unless it is added to it, \code{create*} methods should be provided that
take care of all that (see \fig{lst:ModelExtCreateMethods}).
These create methods should be part of the model to which the \code{Foo}
instance is to be added, in this case \code{ExampleModel}.

\begin{figure}[thb]
  \begin{example}[numbers=left]
public class ExampleModel extends AbstractSBasePlugin {

  // ...

  // only, if ID is not mandatory in Foo
  public Foo createFoo() {
    return createFoo(null);
  }

  public Foo createFoo(String id) {
    Foo foo = new Foo(id, getLevel(), getVersion());
    addFoo(foo);
    return foo;
  }

  public Foo createFoo(String id, int bar) {
    Foo foo = createFoo(id);
    foo.setBar(bar);
    return foo;
  }
}\end{example}
  \caption{Convenience method to create \code{Foo} objects.}
  \label{lst:ModelExtCreateMethods}
\end{figure}


\subsection{The methods \codeNC{equals}, \codeNC{hashCode}, and \codeNC{clone}}

Three more methods should be implemented in an extension
class: \code{equals}, \code{hashCode} and \code{clone}.  This is not
different than when implementing any other Java class, but because mistakes
here can lead to bugs that are very hard to find, we describe the process in
detail.

Whenever two objects \code{o1} and \code{o2} should be regarded as equal,
i.e., all their attributes are equal, the \code{o1.equals(o2)} and the
symmetric case \code{o2.equals(o1)} must return \code{true}, and otherwise
\code{false}. The \code{hashCode} method has two purposes here: allow a
quick check if two objects might be equal, and provide hash values for hash
maps or hash sets and such. The relationship between \code{equals} and
\code{hashCode} is that whenever \code{o1} is equal to \code{o2}, their
hash codes must be the same. Vice versa, whenever their hash codes are
different, they cannot be equal.

\fig{lst:ModelExtEquals} and \fig{lst:ModelExtHashCode} are examples
of how to write these methods for the class \code{Foo} with the attribute
\code{bar}.  Since \code{equals} accepts general objects, it first needs to
check if the passed object is of the same class as the object it is called
on.  Luckily, this has been implemented in \AbstractTreeNode, the super
class of \AbstractSBase. Each class only checks the attributes it adds to
the super class when extending it, but not the \code{ListOf}s, because they
are automatically checked in the \AbstractTreeNode class, the super class
of \AbstractSBase.

\begin{figure}[htb]
  \begin{example}[numbers=left]
@Override
public boolean equals(Object object) {
  boolean equals = super.equals(object);    // recursively checks all children
  if (equals) {
    Foo foo = (Foo) object;
    equals &= foo.isSetBar() == isSetBar();
    if (equals && isSetBar()) {
      // Note: strictly speaking, this is only possible if the return type is some Object. For simple data
      // types, such as boolean, int, short, etc., the corresponding wrapper classes should be called instead
      // or a direct comparison should be performed.
      equals &= (foo.getBar().equals(getBar()));
    }
    // ...
    // further attributes
  }
  return equals;
}\end{example}
  \caption{Example of the \code{equals} method.}
  \label{lst:ModelExtEquals}
\end{figure}

\begin{figure}[htb]
  \begin{example}[numbers=left]
@Override 
public int hashCode() {
  final int prime = 491;              // Some arbitrarily large prime number.
  int hashCode = super.hashCode();    // Recursively checks all children.
  if (isSetBar()) {
    hashCode += prime * getBar().hashCode();
  }
  // ...
  // further attributes

  return hashCode;
}\end{example}
 \caption{Example of the \code{hashCode} method. The variable \code{prime}
   should be a large prime number to  prevent collisions.}
 \label{lst:ModelExtHashCode}
\end{figure}

fig{lst:ModelExtClone} and~\fig{lst:ModelExtCloneFoo} illustrates
implementations of \code{clone()} methods.  To clone an object, the call to
the \code{clone()} method is delegated to a constructor of that class that
takes an instance of itself as argument.  There, all the elements of the
class must be copied, which may require recursive cloning.

Although JSBML defines all SBML elements in a hierarchical data structure, it
is still not possible to recursively clone child elements within the
constructor of some abstract superclasses because these can be of various
types and they cannot simply be organized as a list of children.

\begin{figure}[htb]
  \begin{example}[numbers=left]
@Override public ExampleModel clone() {
  return new ExampleModel(this);
}

public ExampleModel(ExampleModel model) {
  super(model);  // This step is critical!
  // Deep cloning of all elements:
  if (model.isSetListOfFoos()) {
    listOfFoos = model.listOfFoos.clone();
  }
}\end{example}
 \caption{Example of the \code{clone} method for the \code{ExampleModel} class.}
 \label{lst:ModelExtClone}
\end{figure}

\begin{figure}[htb]
  \begin{example}[numbers=left]
@Override public Foo clone() {
  return new Foo(this);
}

public Foo(Foo f) {
  super(f);  // This step is critical!

  // Integer objects are immutable, so it is sufficient to copy the pointer
  bar = f.bar;
}\end{example}
  \caption{Example of the \code{clone} method for the \code{Foo} class.}
  \label{lst:ModelExtCloneFoo}
\end{figure}


