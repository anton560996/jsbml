% -*- TeX-master: "User_guide"; fill-column: 75 -*-

\section{Organizing the source code}
\label{sec:example-organization}

In the JSBML SVN repository, all extensions are found in the subdirectory
named \code{extensions} inside the \code{trunk} directory.  (The process
for checking out a local copy of the repository is described
in \sec{sec:SourceDistribution}.) Each extension is named after the
corresponding SBML short name for the \SBMLthree package; for example,
\code{fbc} for the Flux Balance Constraints package, \code{layout} for the
Layout package, and so on.  The source directories for the extensions
follow some basic conventions for their organization and contents.

\begin{wrapfigure}[22]{r}{2.1in}
  \vspace*{-3.5ex}
  \dirtree{%
    .1 /.
    .2 build.xml.
    .2 doc/.
    .3 img/.
    .2 lib/.
    .2 resources/.
    .2 src/.
    .3 org/.
    .4 sbml/.
    .5 jsbml/.
    .6 ext/.
    .7 NAME/.
    .6 xml/.
    .7 parsers/.
    .2 test/.
  }
  \caption{Typical structure of the source directory for a JSBML 1.0 extension.
    The root of the tree shown here is the \code{extensions/NAME} subdirectory,
    which is located within the \code{trunk} subdirectory of the JSBML SVN
    repository.}
  \label{fig:extension-src-structure}
\end{wrapfigure}
When creating a new extension for JSBML, please follow the conventions used
in the existing extension directories.  These conventions are illustrated in
\fig{fig:extension-src-structure}.  There should be a build script in a file
named ``\code{build.xml}'' for use with Ant~\citep{ApacheAnt}, and several
subdirectories.  The \code{doc} subdirectory should contain documentation
about the extension, preferably with a subdirectory of its own, \code{img},
containing at least a UML diagram of the type hierarchy of the package.  This
can be in the form of, for instance, a Graphviz~\cite{graphvizWebsite} file
\code{type\_hierarchy.dot}, so that the diagram can be generated in different
image formats.  The extension directory should also contain a \code{lib}
subdirectory where any package-specific, third-party libraries are located; a
\code{resources} subdirectory for any non-source files that may be required
by the extension code; an \code{src} subdirectory for the Java source code
comprising the extension; and finally, a \code{test} subdirectory containing
tests for the extension code, preferably in JUnit~\cite{junitWebsite}
format.

Note the structure of the \code{src} subdirectory. A JSBML extension must
define at least two packages: \code{org.sbml.jsbml.ext.NAME}, for the
data structures and code for defining and manipulating the SBML components
specified by the extension, and \code{org.sbml.jsbml.xml.parsers}, for the
code that reads and writes SBML files with the extension
constructs.  Per Java conventions, these source subdirectories are
organized hierarchically based on the package components, which leads to
the nested structure shown in \fig{fig:extension-src-structure}.


