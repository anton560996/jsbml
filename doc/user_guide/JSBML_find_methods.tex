% -*- TeX-master: "User_guide"; fill-column: 75 -*-

\section{The \codeNC{find*} methods}
\label{sec:find-methods}

JSBML provides developers with a number of \code{find*} methods
\index{JSBML!\code{find*} methods}%
on a \Model to help query for elements based on their identifiers or
names. Software can search for various instances of \code{SBase} (for
instance, \CallableSBase, \NamedSBase, and \NamedSBaseWithDerivedUnit);
using methods such as \code{findLocalParameters}, \code{findQuantity},
\code{findQuantityWithUnit}, \code{findSymbol}, and \code{findVariable},
software can also search for the corresponding model element.  They enable
software to work with SBML models more easily, without the need for
explicit separate iteration loops for these common operations.

As of JSBML version 1.0, the \code{find*} methods do no longer query the
model in an iterative way. Instead, the maps described in
\sec{sec:exceptions} are used to access elements based on their \code{id}
attribute. Similarly, the \SBMLDocument can also directly access any of its
subelements for a given \code{metaid}. Such a search can be performed in
logarithmic runtime, i.e., $O(\,\log_2 n)$.

