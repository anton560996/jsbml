% -*- TeX-master: "User_guide"; fill-column: 75 -*-

\section{Logging facilities}
\index{logging}%

JSBML includes the logger provided by the log4j
project~\citep{log4j}.  Log4j allows us to use six levels of logging
(\code{TRACE}, \code{DEBUG}, \code{INFO}, \code{WARN}, \code{ERROR}, and
\code{FATAL}) but internally, JSBML mainly uses \code{ERROR}, \code{WARN},
and \code{DEBUG}. The default configuration of log4j used in JSBML can be
found in the folder \code{resources} with the name \code{log4j.properties}.
In this file, you will find some documentation of which JSBML classes do
some logging and at which levels.
 
If a software package using JSBML does not change the default settings, all
the log messages, starting at the info level (meaning info, warn, error and
fatal), will be printed on the console.  Some of these messages might be
useful to warn end-users that something has gone wrong.


\subsection{Changing the log4j configuration}

If you want to modify the default log4j behavior, you will need to create a
custom log4j configuration file. The best way to do this, as described in
the log4j manual~\citep{log4j}, is to use the environment variable
\code{log4j.configuration} to point to the desired configuration file. One
way to accomplish this is to add the following option to your \code{java}
command (shown here for Unix/Linux and Mac~OS~X, but other operating
systems have analogous facilities):

\begin{example}[style=bash, title={Command line option making log4j
    use a different configuration file.  This syntax applies to Unix-like
    systems.}] 
-Dlog4j.configuration=/home/user/myLog4j.properties
\end{example}


\subsection{Some example configurations}
\ifthenelse{\boolean{includeCodeExample}}{}{\index{logging!configuration}}%

\fig{fig:log4j-simple-example} gives a short example of a log4j
configuration file.  The effect of this particular configuration is to
change the threshold of all loggers in the \code{org.sbml.jsbml.util}
package to \code{DEBUG}, which results in all changes that happen to SBML
elements to be logged. The class \SimpleTreeNodeChangeListener{}  will then
output the old value and the new value whenever a setter method is used on
the SBML elements.

\begin{figure}[t]
  \exampleFile[style=bash, numbers=left, caption={}]{src/log4j_debug_example.properties}
  \caption{A simple log4j configuration example.  This sets the logging
    level of loggers in the \code{org.sbml.jsbml.util} to \code{DEBUG},
    causing all changes to SBML elements to be logged.}
  \label{fig:log4j-simple-example}
\end{figure}

If your application is deployed in a server such as Tomcat~\cite{tomcat},
it may be useful to define a log4j ``appender'' that will send some
messages by email.  \fig{fig:log4j-email-example} gives an example
of doing this.  It configures log4j so that any messages at the
\code{ERROR} level \index{logging} are sent by mail. All the messages are
also written to a rolling log file.

Note that using log4j's alternative, XML-based approach to defining
configurations instead of a properties file, you can configure log4j to
direct some log messages \index{logging}%
to one appender and others to another appender, using the
\code{LevelRange} filter. In this way, it would be possible to cause
\code{DEBUG} messages to be written to a separate file.

\begin{figure}[t]
  \exampleFile[style=bash, numbers=left, caption={}]{src/log4j_email_example.properties}
  \caption{Example of configuring log4j to send email messages for log
    events at the \code{ERROR} level.}
  \label{fig:log4j-email-example}
\end{figure}

Finally, be warned that when you enable the debug level \index{logging} on
some loggers, they may produce copious output.  You may wish to investigate
some of the freely-available software for log
viewing~\cite{logViewersWebpage} to work with the log files.

