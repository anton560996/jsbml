% -*- TeX-master: "User_guide"; fill-column: 75 -*-

\section{Implementing the reader and writer for an SBML package}

One last thing is missing in order to be able to read and write SBML files properly
using the new extension: a \code{ReadingParser} and a \code{WritingParser}. An easy way to provide that is
to extend the \AbstractReaderWriter that extends both interfaces, and then implement some of 
the required methods in a generic way. To
implement the parser, in this case the \code{ExampleParser}, one should
start with two members and two simple methods, as shown in
\fig{lst:ModelExtParserClass}. As this code fragment shows, an additional class
\code{ExampleConstants} and an enum \code{ExampleListType} are used. The class \code{ExampleConstants}
is used to keep track of all the static \String{}s used in the extension so that we are sure 
that the same value is used everywhere. The enum  \code{ExampleListType} can be used to keep
track of which \code{ListOf} we are in while reading an XML file.

\begin{figure}[htb]
  \begin{example}
public class ExampleParser extends AbstractReaderWriter {
  /**
   * The logger for this parser
   */
  private Logger logger = Logger.getLogger(ExampleParser.class);

  /**
   * The ExampleListType enum which represents the name of the list this parser is currently reading.
   */
  private ExampleListType groupList = ExampleListType.none;

  /* (non-Javadoc)
   * @see org.sbml.jsbml.xml.parsers.AbstractReaderWriter#getShortLabel()
   */
  public String getShortLabel() {
    return ExampleConstants.shortLabel;
  }

  /* (non-Javadoc)
   * @see org.sbml.jsbml.xml.parsers.AbstractReaderWriter#getNamespaceURI()
   */
  public String getNamespaceURI() {
    return ExampleConstants.namespaceURI;
  }
}\end{example}
 \caption{The first part of the parser for the extension.}
 \label{lst:ModelExtParserClass}
\end{figure}


\subsection{Reading}

The class \code{AbstractReaderWriter} provides more or less all the 
features needed to read the XML file for your extension---you just need to implement one
method from the \code{Reader} interface. In a future version of JSBML, this method
may be implemented in a generic way using the java reflection API.

The \code{processStartElement()} method is responsible for handling start
elements, such as \code{<listOfFoos>}, and creating the appropriate
objects.  The \code{contextObject} is the object that represents the parent
node of the tag the parser just encountered.  First, you need to check for
every class that may be a parent node of the classes in your extension.  In
this case, those are objects of the classes \Model, \code{Foo} and
\code{ListOf}.  Note that the \code{ExampleModel} has no corresponding XML
tag and the core model is handled by the core parser.  This also
means that the context object of a ListOfFoos is not of the type
\code{ExampleModel}, but of type \Model.  But since the \code{ListOfFoos}
can only be added to an \code{ExampleModel}, the extension is retrieved or
created on the fly.

The \code{groupList} variable keeps track of the current location in
nested structures.  If the \code{listOfFoos} starting tag is encountered,
the corresponding enum value is assigned to that variable.  Due to Java's
type erasure, the context object inside a \code{listOfFoos} tag is of type
\code{ListOf<?>} and a correctly set \code{groupList} variable is the only
way of knowing the current location.  If we have checked that we are, in fact,
inside a \code{listOfFoos} node, and encounter a \code{foo} tag, we create a
\code{Foo} object and add it to the example model.  Technically, it is
added to the \code{ListOfFoos} of the example model, but because
\code{ExampleModel} provides convenience methods for managing its lists, it
is easier to call the \code{addFoo()} method on it.

\begin{figure}[tb]
  \begin{example}
// Create the proper object and link it to its parent.
public Object processStartElement(String elementName, String prefix,
    boolean hasAttributes, boolean hasNamespaces, Object contextObject) {

  if (contextObject instanceof Model) {
    Model model = (Model) contextObject;
    ExampleModel exModel = null;

    if (model.getExtension(ExampleConstants.namespaceURI) != null) {
      exModel = (ExampleModel) model.getExtension(ExampleConstants.namespaceURI);
    } else {
      exModel = new ExampleModel(model);
      model.addExtension(ExampleConstants.namespaceURI, exModel);
    }

    if (elementName.equals("listOfFoos")) {

      ListOf<Foos> listOfFoos = exModel.getListOfFoos();
      this.groupList = ExampleListType.listOfFoos;
      return listOfFoos;
    }
  } else if (contextObject instanceof Foo) {
    Foo foo = (Foo) contextObject;

    // if Foo would have children, that would go here

  }
  else if (contextObject instanceof ListOf<?>) {
    ListOf<SBase> listOf = (ListOf<SBase>) contextObject;

    if (elementName.equals("foo") && this.groupList.equals(ExampleListType.listOfFoos)) {
      Model model = (Model) listOf.getParentSBMLObject();
      ExampleModel exModel = (ExampleModel) model.getExtension(ExampleConstants.namespaceURI);

      Foo foo = new Foo();
      exModel.addFoo(foo);
      return foo;
    }
  }
  return contextObject;
}\end{example}
  \caption{Extension parser: \code{processStartElement()}.}
  \label{lst:ModelExtParserStartElement}
\end{figure}

The \code{processEndElement()} (see \fig{lst:ModelExtParserEndElement}) 
method is called when a closing tag is
encountered.  The \code{groupList} attribute needs to be updated to reflect the step
up in the tree of nested elements.  In this example, if the end of
\code{</listOfFoos>} is reached, we certainly are inside the model tags
again, which is denoted by \emph{none}.  Of course, more complicated
extensions with nested lists will require more elaborate handling, but it
should remain straightforward. If you do not use an enum or something else to
keep track of which \code{ListOf} you are in, and if you do not need to do other
things when a closing XML tag is encountered, you do not need to implement this method.

\begin{figure}[htb]
  \begin{example}
public boolean processEndElement(String elementName, String prefix,
  boolean isNested, Object contextObject) {

  if (elementName.equals("listOfFoos") {
    this.groupList = ExampleListType.none;
  }

  return true;
}\end{example}
  \caption{Extension parser: \code{processEndElement()}.}
  \label{lst:ModelExtParserEndElement}
\end{figure}

The attributes of an XML element are read into the corresponding object via
the \code{readAttributes()} method that must be implemented for each class.
An example is shown in \fig{lst:ModelExtReadAttributes} for the example class
\code{Foo}. The \code{AbstractReaderWriter} will use these methods to set
the attribute values into the java objects.

\begin{figure}[htb]
  \begin{example}
@Override
public boolean readAttribute(String attributeName, String prefix, String value) {

  boolean isAttributeRead = super.readAttribute(attributeName, prefix, value);

  if (!isAttributeRead) {
    isAttributeRead = true;

    if (attributeName.equals(ExampleConstants.bar)) {
      setBar(StringTools.parseSBMLInt(value));
    } else {
      isAttributeRead = false;
    }
  }

  return isAttributeRead;
}\end{example}
  \caption{Method to read the XML attributes.}
  \label{lst:ModelExtReadAttributes}
\end{figure}


\subsection{Writing}

The method \code{getListOfSBMLElementsToWrite()}  must return a list of all
objects that have to be written because of the passed object.  In this way,
the writer can traverse the XML tree to write all nodes. If the classes of the
extension follow the structured advice in \sec{sec:creating-obj-hierarchy}, this
method does not need to be implement and the basic implementation from \code{AbstractReaderWriter}
can be used. This basic implementation makes use of the method \code{TreeNode.children()}
to find the list of children to write.

In some cases, it may be necessary to modify \code{writeElement()}.  For example, this can happen when the same class is mapped to
different XML tags, e.g., a default element and multiple additional tags.
If this would be represented not via an attribute, but by using different
tags, one could alter the name of the XML object in this method.

The actual writing of XML attributes must be implemented in each of the
classes in the \code{writeXMLAttributes()}.  An example is shown in 
\fig{lst:ModelExtCreateXMLAttributes} for the class \code{Foo}.
Then the \code{AbstractReaderWriter} class will use these methods to write
the attributes.

\begin{figure}[htb]
  \vspace*{-1ex}
  \begin{example}
public class Foo extends AbstractNamedSBase {
  ...
  public Map<String, String> writeXMLAttributes() {
    Map<String, String> attributes = super.writeXMLAttributes();
    if (isSetBar()) {
      attributes.remove("bar");
      attributes.put(Foo.shortLabel + ":bar", getBar());

      // Note that in case of double values, the user's locale needs to be taken into account in order to
      // prevent the Writer from numbers in the wrong format. Even in the case of Integer it can be important, 
      // because in some languages very strange number symbols are used (e.g., Thai or Chinese) and hence, the
      // UTF-8 encoding of XML in SBML will lead to SBML files that cannot be parsed again. SBML only accepts 
      // English doubles. Since bar represents an integer, less errors are to be expected.
    }

    // ... Handling of other class attributes ...
  }
}\end{example}
  \caption{Method to write the XML attributes.}
  \label{lst:ModelExtCreateXMLAttributes}
\end{figure}
