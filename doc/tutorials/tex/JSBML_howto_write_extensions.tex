\section{How to implement extensions in JSBML}
\label{sec:howToExtension}

This section presents an example for implementing SBML extensions in JSBML.
For this, we define the \emph{Example} extension specification and use it to explain the necessary steps to implement it in JSBML.

\subsection{Extending an \texttt{SBase}}
\label{subsec:extendingSBase}

In most cases, you probably want to extend a model.
Listing~\vref{lst:ModelExtClass} shows the beginning of the class \texttt{ExampleModel} that is an extension to the standard Model of the SBML core.
\begin{lstlisting}[language=Java,caption={Extending \texttt{AbstractSBasePlugin}},label={lst:ModelExtClass}]
public class ExampleModel extends AbstractSBasePlugin {

  public ExampleModel(Model model) {
    super(model);
  }

}
\end{lstlisting}
Technically, an extension needs to implement the \texttt{SBasePlugin} interface,
but since the abstract class \texttt{AbstractSBasePlugin} already implements some important methods, extending that one should be preferred.

In this example, the constructor accepts an object that is a \texttt{Model}, because that is what we want to extend.
The call to the super constructor will save the given model as the \texttt{SBase} that is being extended in the \texttt{extendedSBase} attribute.
For convenience, a getModel() method to retrieve the extended model should also be added
\begin{lstlisting}[language=Java,caption={Convenience method to retrieve the extended model},label={lst:ModelExtGetModel}]
  public Model getModel() {
    return (Model) getExtendedSBase();
  }
\end{lstlisting}


\subsection{Adding new classes}
\label{subsec:addingClasses}
In almost all cases, extensions introduce new classes that have no counterpart in the SBML core.
Since those new classes are no extensions to existing ones, no extension-specific work has to be done here.
In the \emph{Example} extension, there is the new \texttt{Foo} class that is an \texttt{SBase} and extends \texttt{AbstractNamedSBase}.
It has the three attributes \emph{id}, \emph{name}, and \emph{bar}.
For each attribute, there need to be the following five methods, shown here for the \emph{bar} attribute, which is an integer:
\begin{lstlisting}[language=Java,float,caption={Five necessary methods that should be created for each \texttt{Foo} class attribute},label={lst:ModelExtFooBar}]
  public int getBar();
  public boolean isSetBar();
  public void setBar(int value);
  public boolean unsetBar();

  // if the attribute is a boolean type, additionally
  public boolean isBar();
\end{lstlisting}
Note, that if \emph{bar} would be a boolean type, we should also provide the method \texttt{isBar()}, which delegates to \texttt{getBar()}.

In this special case, \emph{id} and \emph{name} should be unique, so it also implements the \texttt{UniqueNamedSBase} interface.
Because of that, you will be required to implement the above mentioned methods for \emph{id} anyway, those for \emph{name} are already present in the abstract super class.
Listing~\vref{lst:ModelExtFooBarDefault} show how those methods should be implemented in general. It is very important to call the \texttt{FirePropertyChange} listener in the set and unset methods and to throw the \texttt{PropertyUndefineError} in the \texttt{method} if the attribute is not set.

\begin{lstlisting}[language=Java,caption={Five necessary methods that should be created for each \texttt{Foo} class attribute in detail},label={lst:ModelExtFooBarDefault}]
  // use Integer, so we can denote unset values as null
  public Integer bar;

  public int getBar() {
    if (isSetBar()) {
      return bar.intValue();
    }
    throw new PropertyUndefinedError(ExampleConstant.bar, this);
  }

  public boolean isSetBar() {
    return this.bar != null;
  }

  public void setBar(int value) {
    Integer oldBar = this.bar;
    this.bar = bar;
    firePropertyChange(ExampleConstant.bar, oldBar, this.bar);
  }

  public boolean unsetBar() {
    if (isSetBar()) {
      Integer oldBar = this.bar;
      this.bar = null;
      firePropertyChange(ExampleConstant.bar, oldBar, this.bar);
      return true;
    }
    return false;
  }
\end{lstlisting}

Even though some or all of the attributes of a class are mandatory, the default constructor without arguments needs to be defined.
This is due to the internal working of parsers that read SBML files and create the data structure in memory.
All attributes can be set after the object has been created.

Nevertheless, some cases are more frequent than others and one can define constructors for those cases.
On the other hand, creating a separate constructor for each combination of possible passed argument will probably create too many lines of code
that are confusing and more difficult to maintain.

You should at least have the constructors listed in Listing ~\vref{lst:ModelExtFooConstructors}.
As you can see, constructors for id only, level and version only, and all together are implemented.
If you delegate the constructor call to the super class, you have to take care of the initialization of your custom attributes yourself (by calling a method like \texttt{initDefaults()}).
If you delegate to another constructor in your class, you only have to do that at the last one in the delegation chain.
Also, as you can see, this class requires at minimum SBML Level 3, Version 1.

\begin{lstlisting}[language=Java,caption={Constructors for \texttt{Foo}},label={lst:ModelExtFooConstructors}]
  public Foo() {
    super();
    initDefaults();
  }

  public Foo(String id) {
    super(id);
    initDefaults();
  }

  public Foo(int level, int version){
    this(null, null, level, version);
  }

  public Foo(String id, int level, int version) {
    this(id, null, level, version);
  }

  public Foo(String id, String name, int level, int version) {
    super(id, name, level, version);
    if (getLevelAndVersion().compareTo(Integer.valueOf(3), Integer.valueOf(1)) < 0) {
      throw new LevelVersionError(getElementName(), level, version);
    }
    initDefaults();
  }

  /**
   * Clone constructor
   */
  public Foo(Foo foo) {
    super(foo);
    
    bar = foo.bar;
  }

  public void initDefaults() {
    addNamespace(ExampleConstant.namespaceURI);
    bar = null;
  }
\end{lstlisting}

As stated above, you may also have additional constructors like this one:
\begin{lstlisting}[language=Java,caption={Additional constructor for \texttt{Foo}},label={lst:ModelExtFooConstructorsAdditional}]
  public Foo(String id, int bar) {
    this(id);
    setBar(bar);
  }
\end{lstlisting}



\subsection{\texttt{ListOf}s}

The \emph{Example} extension adds no new attributes to the extended model,
but it introduces a new child in form of a list, in this case a \texttt{ListOfFoos},
for the new class \texttt{Foo}.
Instances of \texttt{Foo} can be children of the extended model via a newly defined \texttt{ListOfFoos}.
For this, the methods \texttt{isSetListOfFoos()}, \texttt{getListOfFoos()}, \texttt{setListOfFoos(ListOf<Foo>)}, and \texttt{unsetListOfFoos()} need to be implemented (see Listing ~\vref{lst:ModelExtListOfFoosBasic}).

\begin{lstlisting}[language=Java,caption={Implementation of the ListOf methods: \texttt{isSetListOfFoos()}, \texttt{getListOfFoos()}, \texttt{setListOfFoos()}},label={lst:ModelExtListOfFoosBasic}]
public boolean isSetListOfFoos() {
    if ((listOfFoos == null) || listOfFoos.isEmpty()) {
      return false;
    }
    return true;
  }

  public ListOf<Foo> getListOfFoos() {
    if (!isSetListOfFoos()) {
      Model m = getModel();
      listOfFoos = new ListOf<Foo>(m.getLevel(), m.getVersion());
      listOfFoos.addNamespace(ExampleConstants.namespaceURI);
      listOfFoos.setSBaseListType(ListOf.Type.other);
      m.registerChild(listOfFoos);
    }
    return ListOfFoos;
  }

  public void setListOfFoos(ListOf<Foo> listOfFoos) {
    unsetListOfFoos();
    this.listOfFoos = listOfFoos;
    getModel().registerChild(this.listOfFoos);
  }

  public boolean unsetListOfFoos() {
    if(isSetListOfFoos()) {
      ListOf<Foos> oldFoos = this.listOfFoos;
      this.listOfFoos = null;
      oldFoos.fireNodeRemovedEvent();
      return true;
    }
    return false;
  }
\end{lstlisting}

When adding and removing Foo objects to the model, direct access to the ListOfs should not be necessary.
Therefore, convenience methods for adding and removing an object should be added to the model, which will also do additional consistency checking (Listing ~\vref{lst:ModelExtAddRemoveFoos}).

\begin{lstlisting}[language=Java,caption={Implementation of ListOf methods \texttt{addFoo(Foo foo)}, \texttt{removeFoo(Foo foo)}, \texttt{removeFoo(int i)}},label={lst:ModelExtAddRemoveFoos}]
  public boolean addFoo(Foo foo) {
      return getListOfFoos().add(foo);
  }

  public boolean removeFoo(Foo foo) {
    if (isSetListOfFoos()) {
      return getListOfFoos().remove(foo);
    }
    return false;
  }

  public void removeFoo(int i) {
    if (!isSetListOfFoos()) {
      throw new IndexOutOfBoundsException(Integer.toString(i));
    }
    listOfFoos.remove(i);
  }

  // if the ID is mandatory for Foo objects, one should also add
  public void removeFoo(String id) {
    return getListOfFoos().removeFirst(new NameFilter(id));
  }
\end{lstlisting}

To let the additional \texttt{ListOfFoo} appear as a child of the standard model, the important methods for the \texttt{TreeNode} need to be implemented (see Listing ~\vref{lst:ModelExtChildren}).
\texttt{getAllowsChildren()} should return \texttt{true} in this case, since this extension obviously allows children.
The child count and the indices of the children is a bit more complicated, because it varies with the number of \texttt{ListOf}s that actually contain elements.
So, for every non-empty \texttt{ListOf} child of our model extension, we increase the counter by one.
If a child is queried by its index, the possibility of an index shift needs to be taken into account.

\begin{lstlisting}[language=Java,caption={Methods which need to be implemented to make the children available in the extended model},label={lst:ModelExtChildren}]
  public boolean getAllowsChildren() {
    return true;
  }

  public int getChildCount() {
    int count = 0;

    if (isSetListOfFoos()) {
      count++;
    }
    // same for each additional ListOf* in this extension
    return count;
  }

  public SBase getChildAt(int childIndex) {
    if (childIndex < 0) {
      throw new IndexOutOfBoundsException(childIndex + " < 0");
    }

    int pos = 0;
    if (isSetListOfFoos()) {
      if (pos == childIndex) {
        return getListOfFoos();
      }
      pos++;
    }
    // same for each additional ListOf* in this extension
    throw new IndexOutOfBoundsException(MessageFormat.format("Index {0,number,integer} >= {1,number,integer}", childIndex, +((int) Math.min(pos, 0))));
  }
\end{lstlisting}



\subsection{Create methods}

Because a newly created instance of type Foo is not part of the model unless it is added to it,
\texttt{create*} methods should be provided that take care of all that (see Listing ~\ref{lst:ModelExtCreateMethods}). 
These create methods should be part of the model to which the Foo instance should be added, in this case ExampleModel.

\begin{lstlisting}[language=Java,caption={Convenience method to create elements},label={lst:ModelExtCreateMethods}]
public class ExampleModel extends AbstractSBasePlugin {

  ...

  // only, if ID is not mandatory in Foo
  public Foo createFoo() {
    return createFoo(null);
  }

  public Foo createFoo(String id) {
    Foo foo = new Foo(id, getModel().getLevel(), getModel().getVersion());
    addFoo(foo);
    return foo;
  }

  public Foo createFoo(String id, int bar) {
    Foo foo = createFoo(id);
    foo.setBar(bar);
    return foo;
  }
}
\end{lstlisting}


\subsection{\texttt{equals}, \texttt{hashCode}, and \texttt{clone}}
There are three further methods which should be implemented in an extension class: \texttt{equals}, \texttt{hashCode} and \texttt{clone}.
This is no different than in any other Java class, but since mistakes here can lead to bugs that are very hard to find, we will go through the basics anyway.

Whenever two objects \texttt{o1} and \texttt{o2} should be regarded as equal, i.e., all their attributes are equal, the \texttt{o1.equals(o2)} and the symmetric case \texttt{o2.equals(o1)} must return \texttt{true}, and otherwise \texttt{false}.
The \texttt{hashCode} method has two purposes here, namely allowing for a quick check if two objects might be equal or not, and providing hash values for hash maps or hash sets and such.
The relationship between \texttt{equals} and \texttt{hashCode} is that whenever \texttt{o1} is equal to \texttt{o2}, their hash codes must be the same.
Vice versa, whenever their hash codes are different, they cannot be equal.

Listing \ref{lst:ModelExtEquals} and \ref{lst:ModelExtHashCode} are examples how to write these methods for the class \texttt{Foo} with the attribute \texttt{bar}.
Since \texttt{equals} accepts general objects, it first needs to check if the passed object is of the same class as the object it is called on.
Luckily, this has been implemented in \texttt{AbstractTreeNode}, the super class of \texttt{AbstractSBase}.
Each class only checks the attributes it adds to the super class when extending it, but not the \texttt{ListOf}s,
because they are automatically checked in the \texttt{AbstractTreeNode} class, the super class of \texttt{AbstractSBase}.

\begin{lstlisting}[language=Java,caption={Example of the \texttt{equals} method},label={lst:ModelExtEquals}]
@Override
  public boolean equals(Object object) {
    boolean equals = super.equals(object);    // recursively checks all children
    if (equals) {
      Foo foo = (Foo) object;
      equals &= foo.isSetBar() == isSetBar();
      if (equals && isSetBar()) {
        equals &= (foo.getBar().equals(getBar()));
      }
      // ...
      // further attributes
    }
    return equals;
  }
\end{lstlisting}


\begin{lstlisting}[language=Java,caption={Example of the \texttt{hashCode} method. The variable \texttt{prime} should be a big prime number to prevent collisions},label={lst:ModelExtHashCode}]
  @Override
  public int hashCode() {
    final int prime = 491;
    int hashCode = super.hashCode();    // recursively checks all children
    if (isSetBar()) {
      hashCode += prime * getBar().hashCode();
    }
    // ...
    // further attributes
    
    return hashCode;
  }
\end{lstlisting}

To clone an object, the call to the \texttt{clone()} method is delegated to a constructor of that class that takes an object of itself as argument.
There, all the elements of the class must be copied, which may require recursive cloning.

\begin{lstlisting}[language=Java,caption={Example of the \texttt{clone} method for the \texttt{ExampleModel} class},label={lst:ModelExtClone}]
  public ExampleModel clone() {
    return new ExampleModel(this);
  }

  public ExampleModel(ExampleModel model) {
    super();

    if (model.isSetListOfFoos()) {
      listOfFoos = model.listOfFoos.clone();
    }
  }
\end{lstlisting}

\begin{lstlisting}[language=Java,caption={Example of the \texttt{clone} method for the \texttt{Foo} class},label={lst:ModelExtCloneFoo}]
  public Foo clone() {
    return new Foo(this);
  }

  public Foo(Foo f) {
    super();

    // Integer objects are immutable, so they can be copied
    bar = f.bar;
  }
\end{lstlisting}


\subsection{Writing the parser/writer}

One last thing is missing to be able to properly read and write SBML files using the new extension: a parser and a writer.
An easy way to do that is to extend the \texttt{AbstractReaderWriter} and implement the required methods.
To implement the parser, in this case the \texttt{ExamplePaser}, one should start with two members and two simple methods, as shown in Listing \ref{lst:ModelExtParserClass}.

As can be seen from this code snippet, an additional class \texttt{ExampleConstant} and an enum \texttt{ExampleList} are used.

TODO


\begin{lstlisting}[language=Java,caption={The first part of the parser for the extension},label={lst:ModelExtParserClass}]
public class ExampleParser extends AbstractReaderWriter {

  /**
   * The logger for this parser
   */
  private Logger logger = Logger.getLogger(ExampleParser.class);
  
  /**
   * The ExampleList enum which represents the name of the list this parser is
   * currently reading.
   */
  private ExampleList groupList = ExampleList.none;
  
  /* (non-Javadoc)
   * @see org.sbml.jsbml.xml.parsers.AbstractReaderWriter#getShortLabel()
   */
  public String getShortLabel() {
    return ExampleConstant.shortLabel;
  }

  /* (non-Javadoc)
   * @see org.sbml.jsbml.xml.parsers.AbstractReaderWriter#getNamespaceURI()
   */
  public String getNamespaceURI() {
    return ExampleConstant.namespaceURI;
  }

}
\end{lstlisting}


\subsubsection{Writing}

The method \texttt{getListOfSBMLElementsToWrite()} (see \ref{lst:ModelExtParserListSBMLToWrite}) has to return a list of all objects that have to be written because of the passed object.
This way, the writer can traverse the XML tree to write all nodes.
Basically, there are three classes of objects that need to be distinguished:
\begin{itemize}
 \item \texttt{SBMLDocument}
 \item extended classes
 \item \texttt{TreeNode}
\end{itemize}
TODO: SBMLDocument.
After that we need to check if the current object is extendable using our extension.
In our example extension, a model can be extended using ExampleModel to also have a list of Foos as children.
In Java, this ListOfFoos is not a children of the original model, but of the example model.
The example model, on the other hand, is just an SBasePlugin, which is not an SBase and also not a children of the original model.
To ``inject'' the ListOfFoos in the right place, all children of the example model in Java become direct children of the original model in XML.

All other objects that implement SBase also implement TreeNode, so we just add all of their children to the list of elements to write.

\begin{lstlisting}[language=Java,caption={Extension parser: \texttt{getListOfSBMLElementsToWrite()}},label={lst:ModelExtParserListSBMLToWrite}]
  public ArrayList<Object> getListOfSBMLElementsToWrite(Object sbase) {

    if (logger.isDebugEnabled()) {
      logger.debug("getListOfSBMLElementsToWrite : " + sbase.getClass().getCanonicalName());
    }
    
    ArrayList<Object> listOfElementsToWrite = new ArrayList<Object>();

    if (sbase instanceof SBMLDocument) {
      // nothing to do
      // TODO : the 'required' attribute is written even if there is no plugin class for the SBMLDocument, so I am not totally sure how this is done.
    } 
    else if (sbase instanceof Model) {
      ExampleModel modelGE = (ExampleModel) ((Model) sbase).getExtension(ExampleConstant.namespaceURI);
      
      Enumeration<TreeNode> children = modelGE.children();
      
      while (children.hasMoreElements()) {
        listOfElementsToWrite.add(children.nextElement());
      }           
    } 
    else if (sbase instanceof TreeNode) {
      Enumeration<TreeNode> children = ((TreeNode) sbase).children();
      
      while (children.hasMoreElements()) {
        listOfElementsToWrite.add(children.nextElement());
      }
    }
    
    if (listOfElementsToWrite.isEmpty()) {
      listOfElementsToWrite = null;
    } else if (logger.isDebugEnabled()) {
      logger.debug("getListOfSBMLElementsToWrite size = " + listOfElementsToWrite.size());
    }

    return listOfElementsToWrite;
  }
\end{lstlisting}

In some cases it may be necessary to modify the \texttt{writeElement()} method.
For example, this can happen when the same Java class is mapped to different XML tags, e.g., a default element and multiple additional tags.
If this would be represented not via an attribute, but by using different tags, one could alter the name of the XML object in this method.

The actual writing of XML attributes must be implemented in each of the classes in the \texttt{writeXMLAttributes()}.
An example is shown in Listing \ref{lst:ModelExtCreateXMLAttributes} for the class \texttt{Foo}.

\begin{lstlisting}[language=Java,caption={Method to write the XML attributes},label={lst:ModelExtCreateXMLAttributes}]
public class Foo extends AbstractNamedSBase {
  ...

  public Map<String, String> writeXMLAttributes() {
    Map<String, String> attributes = super.writeXMLAttributes();
    if (isSetBar()) {
      attributes.remove("bar");
      attributes.put(Foo.shortLabel + ":bar", getBar());
    }
    
    // ...
    // further class attributes
  }
}
\end{lstlisting}


\subsubsection{Parsing}

The \texttt{processStartElement()} method is responsible for handling start elements, such as \texttt{<listOfFoos>}, and creating the appropriate objects.
The \texttt{contextObject} is the object representing the parent node of the tag the parser just encountered.
First, you need to check for every class that may be a parent node of the classes in your extension.
In this case, those are objects of the classes \texttt{Model}, \texttt{Foo} and \texttt{ListOf}.
Note, that the \texttt{ExampleModel} has no corresponding XML tag and the core model is already handled by the core parser.
This also means that the context object of a ListOfFoos is not of the type \texttt{ExampleModel}, but of type \texttt{Model}.
But since the \texttt{ListOfFoos} can only be added to an \texttt{ExampleModel}, the extension is retrieved or created on the fly.

The \texttt{groupList} variable is used to keep track of where we are in nested structures.
If the \texttt{listOfFoos} starting tag is encountered, the corresponding enum value is assigned to that variable.
Due to Java's type erasure, the context object inside a listOfFoos tag is of type \texttt{ListOf<?>} and a correctly set \texttt{groupList} variable is the only way of knowing where we are.
If we have checked that we are, in fact, inside a \texttt{listOfFoos} node and encounter a \texttt{foo} tag, we create \texttt{Foo} object and add it to the example model.
Technically, it is added to the \texttt{ListOfFoos} of the example model, but since \texttt{ExampleModel} provides convenience methods for managing its lists, it is easier to call the \texttt{addFoo()} method on it.

\begin{lstlisting}[language=Java,caption={Extension parser: \texttt{processStartElement()}},label={lst:ModelExtParserStartElement}]
  // Create the proper object and link it to his parent.
  public Object processStartElement(String elementName, String prefix,
      boolean hasAttributes, boolean hasNamespaces, Object contextObject) 
  {
    
    if (contextObject instanceof Model) {
      Model model = (Model) contextObject;
      ExampleModel exModel = null;
      
      if (model.getExtension(ExampleConstant.namespaceURI) != null) {
        exModel = (ExampleModel) model.getExtension(ExampleConstant.namespaceURI);
      } else {
        exModel = new ExampleModel(model);
        model.addExtension(ExampleConstant.namespaceURI, exModel);
      }

      if (elementName.equals("listOfFoos")) {
          
        ListOf<Foos> listOfFoos = exModel.getListOfFoos();
        this.groupList = QualList.listOfFoos;
        return listOfFoos;
      } 
    } else if (contextObject instanceof Foo) {
      Foo foo = (Foo) contextObject;

      // if Foo would have children, that would go here

    }
    else if (contextObject instanceof ListOf<?>) 
    {
      ListOf<SBase> listOf = (ListOf<SBase>) contextObject;

      if (elementName.equals("foo") && this.groupList.equals(QualList.listOfFoos)) { 
        Model model = (Model) listOf.getParentSBMLObject();
        ExampleModel exModel = (ExampleModel) model.getExtension(ExampleConstant.namespaceURI); 
        
        Foo foo = new Foo();       
        exModel.addFoo(foo);
        return foo;
      } 
    }
    return contextObject;
  }
\end{lstlisting}

The processEndElement() method is called whenever a closing tag is encountered.
The groupList attribute needs to be updated to reflect the step up in the tree of nested elements.
In this example, if the end of \texttt{</listOfFoos>} is reached, we certainly are inside the model tags again, which is denoted by \emph{none}.
Of course, more complicated extensions with lots of nested lists need a more elaborate handling here, but it should still be straight-forward.

\begin{lstlisting}[language=Java,caption={Extension parser: \texttt{processEndElement()}},label={lst:ModelExtParserEndElement}]
  /* (non-Javadoc)
   * @see org.sbml.jsbml.xml.parsers.AbstractReaderWriter#processEndElement(java.lang.String, java.lang.String, boolean, java.lang.Object)
   */
  public boolean processEndElement(String elementName, String prefix,
      boolean isNested, Object contextObject) 
  {

    if (elementName.equals("listOfFoos")
    {
      this.groupList = QualList.none;
    }

    return true;
  }
\end{lstlisting}

Attributes of a tag are read into the corresponding object via the \texttt{readAttributes()} method that must be implemented for each class.
An example is shown in Listing \ref{lst:ModelExtReadAttributes} for the class \texttt{Foo}.

\begin{lstlisting}[language=Java,caption={Method to read the XML attributes},label={lst:ModelExtReadAttributes}]
  @Override
  public boolean readAttribute(String attributeName, String prefix, String value) {

    boolean isAttributeRead = super.readAttribute(attributeName, prefix, value);

    if (!isAttributeRead) {
      isAttributeRead = true;

      if (attributeName.equals(ExampleConstant.bar)) {
        setBar(StringTools.parseSBMLInt(value));
      } else {
        isAttributeRead = false;
      }
    }

    return isAttributeRead;
  }
\end{lstlisting}



\section{Implementation checklist}
    \begin{itemize}
        \item [$\Box$] Added the extension to an existing model (see Listing \ref{lst:ModelExtClass})
        \item [$\Box$] Added the five necessary methods for each class attribute (see Listing \ref{lst:ModelExtFooBar}, \ref{lst:ModelExtFooBarDefault}):
            \begin{itemize}
              \item [$\Box$] \texttt{getBar()}
              \item [$\Box$] \texttt{isBarMandatory()}
              \item [$\Box$] \texttt{isBarFoo()}
              \item [$\Box$] \texttt{setBar(int value)}
              \item [$\Box$] \texttt{unsetBar()}
            \end{itemize}
        \item [$\Box$] Added the default constructors (see Listing \ref{lst:ModelExtFooConstructors})
        \item [$\Box$] If the class has children, check if all list methods are implemented (see Listing \ref{lst:ModelExtChildren}, \ref{lst:ModelExtListOfFoosBasic}, \ref{lst:ModelExtAddRemoveFoos}, \ref{lst:ModelExtChildren}):
            \begin{itemize}
              \item [$\Box$] \texttt{isSetListOfFoos()}
              \item [$\Box$] \texttt{getListOfFoos()}
              \item [$\Box$] \texttt{setListOfFoos(ListOf<Foo> listOfFoos)}
              \item [$\Box$] \texttt{addFoo(Foo foo)}
              \item [$\Box$] \texttt{removeFoo(Foo foo)}
              \item [$\Box$] \texttt{removeFoo(int i)}
              \item [$\Box$] \texttt{getAllowsChildren()}
              \item [$\Box$] \texttt{getChildCount()}
            \end{itemize}
        \item [$\Box$] Are all necessary create methods implemented (see Listing \ref{lst:ModelExtCreateMethods})
        \item [$\Box$] Implemented the \texttt{equals} method (see Listing \ref{lst:ModelExtEquals})
        \item [$\Box$] Implemented the \texttt{hashCode} method (see Listing \ref{lst:ModelExtHashCode})
        \item [$\Box$] Implemented the \texttt{clone} method (see Listing \ref{lst:ModelExtClone} and \ref{lst:ModelExtCloneFoo})
        \item [$\Box$] Implemented the \texttt{writeXMLAttribute()} method (see Listing \ref{lst:ModelExtCreateXMLAttributes})
        \item [$\Box$] Implemented the parser/writer method (see Listing \ref{lst:ModelExtParserClass}, \ref{lst:ModelExtParserListSBMLToWrite}, \ref{lst:ModelExtParserStartElement}, \ref{lst:ModelExtParserEndElement}):
   \end{itemize}
