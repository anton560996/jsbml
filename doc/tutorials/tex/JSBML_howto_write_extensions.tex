\section{How to implement extensions in JSBML}
\label{sec:howToExtension}

This section presents an example for implementing SBML extensions in JSBML.
For this, we define the \emph{Example} extension specification and use it to explain the necessary steps to implement it in JSBML.

\subsection{Extending an \texttt{SBase}}
\label{subsec:extendingSBase}

In most cases, you probably want to extend a model.
Listing~\vref{lst:ModelExtClass} shows the beginning of the class \texttt{ExampleModel} that is an extension to the standard Model of the SBML core.
\begin{lstlisting}[language=Java,caption={Extending \texttt{AbstractSBasePlugin}},label={lst:ModelExtClass}]
public class ExampleMode extends AbstractSBasePlugin {

  public ExampleModel(Model model) {
    super(model);
  }

}
\end{lstlisting}
Technically, an extension needs to implement the \texttt{SBasePlugin} interface,
but since the abstract class \texttt{AbstractSBasePlugin} already implements some important methods, extending that one should be preferred.

In this example, the constructor accepts an object that is a \texttt{Model}, because that is what we want to extend.
The call to the super constructor will save the given model as the \texttt{SBase} that is being extended in the \texttt{extendedSBase} attribute.
For convenience, a getModel() method to retrieve the extended model should also be added
\begin{lstlisting}[language=Java,caption={Convenience method to retrieve the extended model},label={lst:ModelExtGetModel}]
  public Model getModel() {
    return (Model)extendedSBase;
  }
\end{lstlisting}


\subsection{Adding new classes}
\label{subsec:addingClasses}
In almost all cases, extensions introduce new classes.
Since those new classes are no extensions to existing ones, no extension-specific work has to be done here.
In the \emph{Example} extension, there is the new \texttt{Foo} class that is an \texttt{SBase} and extends \texttt{AbstractNamedSBase}.
It has the three attributes \emph{id}, \emph{name}, and \emph{bar}.
For each attribute, there need to be the following five methods, shown here for the \emph{bar} attribute, which is an integer:
\begin{lstlisting}[language=Java,float,caption={Five necessary methods that should be created for each \texttt{Foo} class attribute},label={lst:ModelExtFooBar}]
  public int getBar();
  public boolean isBarMandatory();
  public boolean isSetBar();
  public void setBar(int value);
  public boolean unsetBar();
\end{lstlisting}

In this special case, \emph{id} and \emph{name} should be unique, so it also implements the \texttt{UniqueNamedSBase} interface.
Because of that, you will be required to implement the above mentioned methods for \emph{id} anyway, those for \emph{name} are already present in the abstract super class.
Listing~\vref{lst:ModelExtFooBarDefault} show how those methods should be implemented in general. It is very important to call the \texttt{FirePropertyChange} listener in the set and unset methods and to throw the \texttt{PropertyUndefineError} in the \texttt{method}, if the attribute is not set.

\begin{lstlisting}[language=Java,caption={Five necessary methods that should be created for each \texttt{Foo} class attribute in detail},label={lst:ModelExtFooBarDefault}]
  // use Integer, so we can denote unset values as null
  public Integer bar;

  public int getBar() {
    if (isSetBar()) {
      return bar.intValue();
    }
    throw new PropertyUndefinedError(ExampleConstant.bar, this);
  }

  public boolean isBarMandatory() {
    return true;
  }

  public boolean isSetBar() {
    return this.bar != null;
  }

  public void setBar(int value) {
    Integer oldBar = this.bar;
    this.bar = bar;
    firePropertyChange(ExampleConstant.bar, oldBar, this.bar);
  }

  public boolean unsetBar() {
    if (isSetBar()) {
      Integer oldBar = this.bar;
      this.bar = null;
      firePropertyChange(ExampleConstant.bar, oldBar, this.bar);
      return true;
    }
    return false;
  }
\end{lstlisting}

Even though some or all of the attributes of a class are mandatory, the default constructor without arguments needs to be defined.
This is due to the internal working of parsers that read SBML files and create the data structure in memory.
All attributes can be set after the object has been created.

Nevertheless, some cases are more frequent than other and one can define constructors for those cases.
On the other hand, creating a separate constructor for each combination of possible passed argument will probably create to many lines of code
that are confusing and more difficult to maintain.

You should at least have the constructors listed in Listing ~\vref{lst:ModelExtFooConstructors}.
As you can see, constructors for id only, level and version only, and all together are implemented.
If you delegate the constructor call to the super class, you have to take care of the initialization of your custom attributes yourself (by calling a method like \texttt{initDefaults()}).
If you delegate to another constructor in your class, you only have to do that at the last one in the delegation chain.
Also, as you can see, this class requires at minimum SBML level 3, version 1.

\begin{lstlisting}[language=Java,caption={Constructors for \texttt{Foo}},label={lst:ModelExtFooConstructors}]
  public Foo() {
    super();
    initDefaults();
  }

  public Foo(String id) {
    super(id);
    initDefaults();
  }

  public Foo(int level, int version){
    this(null, null, level, version);
  }

  public Foo(String id, int level, int version) {
    this(id, null, level, version);
  }

  public Foo(String id, String name, int level, int version) {
    super(id, name, level, version);
    if (getLevelAndVersion().compareTo(Integer.valueOf(3), Integer.valueOf(1)) < 0) {
      throw new LevelVersionError(getElementName(), level, version);
    }
    initDefaults();
  }

  public void initDefaults() {
    addNamespace(ExampleConstant.namespaceURI);
    bar = null;
  }
\end{lstlisting}

As stated above, you may also have additional constructors like this one:
\begin{lstlisting}[language=Java,caption={Additional constructor for \texttt{Foo}},label={lst:ModelExtFooConstructorsAdditional}]
  public Foo(String id, int bar) {
    this(id);
    setBar(bar);
  }
\end{lstlisting}



\subsection{\texttt{ListOf}s}

The \emph{Example} extension adds no new attributes to the extended model,
but it introduces a new child in form of a list, in this case a \texttt{ListOfFoos},
for the new class \texttt{Foo}.
Instances of \texttt{Foo} can be children of the extended model via a newly defined \texttt{ListOfFoos}.
For this, the methods \texttt{isSetListOfFoos()}, \texttt{getListOfFoos()}, \texttt{setListOfFoos(ListOf<Foo>)}, and \texttt{unsetListOfFoos()} need to be implemented.

\begin{lstlisting}[language=Java,caption={Implementation of ListOf the methods: \texttt{isSetListOfFoos()}, \texttt{getListOfFoos()}, \texttt{setListOfFoos()}},label={lst:ModelExtListOfFoosBasic}]
public boolean isSetListOfFoos() {
    if ((listOfFoos == null) || listOfFoos.isEmpty()) {
      return false;
    }
    return true;
  }

  public ListOf<Transition> getListOfFoos() {
    if (!isSetListOfFoos()) {
      listOfFoos = new ListOf<Foo>(getModel().getLevel(), getModel().getVersion());
      listOfFoos.addNamespace(ExampleConstants.namespaceURI);
    getModel().registerChild(listOfFoos);
    listOfFoos.setSBaseListType(ListOf.Type.other);

    }
    return ListOfFoos;
  }

  public void setListOfFoos(ListOf<Foo> listOfFoos) {
    unsetListOfFoos();
    this.listOfFoos = listOfFoos;
    getModel().registerChild(this.listOfFoos);
  }

  public boolean unsetListOfFoos(){
    if(isSetListOfFoos()) {
      ListOf<Foos> oldFoos = this.listOfFoos;
      this.listOfFoos = null;
      oldFoos.fireNodeRemovedEvent();
      return true;
    }
    return false;
  }
\end{lstlisting}

When adding and removing Foo objects to the model, direct access to the ListOfs should not be necessary.
Therefore, convenience methods for adding and removing object should be added to the model, which will also do additional consistency checking.

\begin{lstlisting}[language=Java,caption={Implementaion of ListOf methods \texttt{addFoo(Foo foo)}, \texttt{removeFoo(Foo foo)}, \texttt{removeFoo(int i)}},label={lst:ModelExtAddRemoveFoos}]
  public boolean addFoo(Foo foo) {
      return getListOfFoos().add(foo);
  }

  public boolean removeFoo(Foo foo) {
    if (isSetListOfFoos()) {
      return listOfFoos.remove(foo);
    }
    return false;
  }

  public void removeFoo(int i) {
    if (!isSetListOfFoos()) {
      throw new IndexOutOfBoundsException(Integer.toString(i));
    }
    listOfFoos.remove(i);
  }
\end{lstlisting}

To let the additional \texttt{ListOfFoo} appear as a child of the standard model, the important methods for the \texttt{JTree} need to be implemented.
\texttt{getAllowsChildren()} should return \texttt{true} in this case, since this extension obviously allows children.
The child count and the indices of the children is bit more complicated, because it varies with the number of \texttt{ListOf}s that actually contain elements.
So, for every non-empty \texttt{ListOf} child of our model extension, we increase the counter by one.
If a child is queried by its index, the possibility of an index shift needs to be taken into account.

\begin{lstlisting}[language=Java,caption={Methods which need to be implemented to make the children available in the extended model},label={lst:ModelExtChildren}]
  public boolean getAllowsChildren() {
    return true;
  }

  public int getChildCount() {
    int count = 0;

    if (isSetListOfFoos()) {
      count++;
    }
    // same for each additional ListOf* in this extension
    return count;
  }

  public SBase getChildAt(int childIndex) {
    if (childIndex < 0) {
      throw new IndexOutOfBoundsException(childIndex + " < 0");
    }

    int pos = 0;
    if (isSetListOfFoos()) {
      if (pos == childIndex) {
        return getListOfFoos();
      }
      pos++;
    }
    // same for each additional ListOf* in this extension
    throw new IndexOutOfBoundsException(String.format("Index %d >= %d", childIndex, +((int) Math.min(pos, 0))));
  }
\end{lstlisting}



\subsection{Create methods}

Because a newly created instance of type Foo is not part of the model unless it is added,
\texttt{create*} methods should be provided that take care of all that.
These create methods should be part of the model to which the Foo instance should be added, in this case ExampleModel.

\begin{lstlisting}[language=Java,caption={Convenience method to retrieve the extended model},label={lst:ModelExtCreateMethods}]
public class ExampleMode extends AbstractSBasePlugin {

  ...

  public Foo createFoo() {
    return createFoo(null);
  }

  public Foo createFoo(String id) {
    Foo foo = new Foo(id);
    addFoo(foo);
    return foo;
  }

  public Foo createFoo(String id, int bar) {
    Foo foo = new QualitativeSpecies(id,
      getModel().getLevel(), getModel().getVersion());
    foo.setBar(bar);
    addFoo(foo);
    return foo;
  }
}
\end{lstlisting}


\subsection{\texttt{equals}, \texttt{hashCode} and \texttt{clone}}
There are three further methods which should be written in an extension class: \texttt{equals}, \texttt{hashCode} and \texttt{clone}. In the Listing \ref{lst:ModelExtEquals}, \ref{lst:ModelExtHashCode}, and \ref{lst:ModelExtClone} are examples how to write these methods for the class \texttt{Foo} with the attribute \texttt{bar}.

\begin{lstlisting}[language=Java,caption={Example of the \texttt{equals} method},label={lst:ModelExtEquals}]
@Override
  public boolean equals(Object object) {
    boolean equals = super.equals(object);
    if (equals) {
      Foo foo = (Foo) object;
      equals &= foo.isSetBar() == isSetBar();
      if (equals && isSetBar()) {
        equals &= (foo.getBar().equals(getBar()));
      }
      // ...
      // further attributes
    }
    return equals;
  }
\end{lstlisting}


\begin{lstlisting}[language=Java,caption={Example of the \texttt{hashCode} method. The variable \texttt{prime} should be a big prime number to prevent collisions},label={lst:ModelExtHashCode}]
  @Override
  public int hashCode() {
    final int prime = 491;
    int hashCode = super.hashCode();
    if (isSetBar()) {
      hashCode += prime * getBar().hashCode();
    }
    // ...
    // further attributes
    
    return hashCode;
  }
\end{lstlisting}

\begin{lstlisting}[language=Java,caption={Example of the \texttt{clone} method},label={lst:ModelExtClone}]
  public AbstractSBase clone() {
    return new Foo(this);
  }
\end{lstlisting}


\subsection{\texttt{writeXMLAttributes}}
To write the xml attributes of each class of the extensions in the SBML document the \texttt{writeXMLAttributes()} method must be written in each class. An example is shown in Listing \ref{lst:ModelExtCreateXMLAttributes}.

\begin{lstlisting}[language=Java,caption={Method to create the XML attributes},label={lst:ModelExtCreateXMLAttributes}]
  public Map<String, String> writeXMLAttributes() {
    Map<String, String> attributes = super.writeXMLAttributes();
    if (isSetBar()) {
      attributes.remove("bar");
      attributes.put(Foo.shortLabel + ":bar", getBar());
    }
    
    // ...
    // further class attributes
  }
\end{lstlisting}



\section{Implementation checklist}
    \begin{itemize}
        \item [$\Box$] Added the extension to an existing model (see Listing \ref{lst:ModelExtClass})
        \item [$\Box$] Added the five necessary methods for each class attribute (see Listing \ref{lst:ModelExtFooBar}, \ref{lst:ModelExtFooBarDefault}):
            \begin{itemize}
              \item [$\Box$] \texttt{getBar()}
              \item [$\Box$] \texttt{isBarMandatory()}
              \item [$\Box$] \texttt{isBarFoo()}
              \item [$\Box$] \texttt{setBar(int value)}
              \item [$\Box$] \texttt{unsetBar()}
            \end{itemize}
        \item [$\Box$] Added the default constructors (see Listing \ref{lst:ModelExtFooConstructors})
        \item [$\Box$] If the class has children are all list methods implemented (see Listing \ref{lst:ModelExtChildren}, \ref{lst:ModelExtListOfFoosBasic}, \ref{lst:ModelExtAddRemoveFoos}, \ref{lst:ModelExtChildren}):
            \begin{itemize}
              \item [$\Box$] \texttt{isSetListOfFoos()}
              \item [$\Box$] \texttt{getSetListOfFoos()}
              \item [$\Box$] \texttt{setSetListOfFoos(ListOf<Foo> listOfFoos)}
              \item [$\Box$] \texttt{addFoo(Foo foo)}
              \item [$\Box$] \texttt{removeFoo(Foo foo)}
              \item [$\Box$] \texttt{removeFoo(int i)}
              \item [$\Box$] \texttt{getAllowsChildren()}
              \item [$\Box$] \texttt{getChildCount()}
            \end{itemize}
        \item [$\Box$] Are all necessary create methods implemented (see Listing \ref{lst:ModelExtCreateMethods})
        \item [$\Box$] Implemented the \texttt{equals} method (see Listing \ref{lst:ModelExtEquals})
        \item [$\Box$] Implemented the \texttt{hashCode} method (see Listing \ref{lst:ModelExtHashCode})
        \item [$\Box$] Implemented the \texttt{clone} method (see Listing \ref{lst:ModelExtClone})
        \item [$\Box$] Implemented the \texttt{writeXMLAttribute()} method (see Listing \ref{lst:ModelExtCreateXMLAttributes})
   \end{itemize}
