\section{Introduction}

JSBML is a library that will help you to manipulate SBML files
\citep{Draeger2011, Draeger2011b}. If you are not familiar with SBML, a good
starting point would be to read the latest SBML
specification\index{SBML!specification}\footnote{\url{%
http://sbml.org/Documents/Specifications/}} \citep{Hucka2010a}. If you have some
other questions about SBML, you may find the answer in the SBML
FAQ\footnote{\url{http://sbml.org/Documents/FAQ}}. JSBML is written in Java\TTra. To
use it, you will need a Java Runtime Environment (JRE) 1.5 or higher. See, for
example, the Java SE download page\footnote{\url{%
http://www.oracle.com/technetwork/java/javase/downloads/index.html}
\label{fn:jvmldl}}.
JSBML also provides several modules. Two of them should ease developers to
interact with CellDesigner or libSBML and one module eases switching from
libSBML to JSBML or the other way around.


\section{Obtaining and setting up JSBML}

\subsection{Using the JSBML JAR file distribution}
Before starting to use JSBML, you will need to configure your class path. JSBML
provides two versions of the JAR file:
\begin{enumerate}
\item including all dependencies - it is sufficient to include just this file in
      your class path.
\item without any dependencies - you need to take care of all the dependencies of
      JSBML by yourself.
\end{enumerate}

The JSBML JAR file with dependencies is a merged JAR file that includes all of
its required third-party libraries. In this case, it is sufficient to include
it into your build or class path in order to use JSBML.



\subsubsection{Dependencies}
\index{JSBML!dependencies}%

When using the JSBML JAR file without dependencies, you need the JSBML
dependencies in addition to the JSBML library. The following list gives you an
overview of all these libraries:
\begin{description}
\item[biojava-1.7-ontology.jar] This is a stripped down version of the
biojava-1.7\footnote{\url{http://biojava.org}} containing mostly
ontology-related classes \citep{Holland2008}.
\item[junit-4.8.jar] This library is only needed, if you want to run the
JUnit\footnote{\url{http://www.junit.org}\label{fn:junit}} tests of JSBML
(located in the \texttt{test} folder).
\item[stax2-api-3.0.3.jar] Used to read and write the XML
files\footnote{\url{http://docs.codehaus.org/display/WSTX/StAX2}}.
\item[stax-api-1.0.1.jar] Used to read and write the XML
files\footnote{\url{http://stax.codehaus.org}}.
\item[woodstox-core-lgpl-4.0.9.jar] Used to read and write the XML files. This
is the actual stax parser implementation JSBML
uses\footnote{\url{http://woodstox.codehaus.org}}.
\item[staxmate-2.0.0.jar] Used to read and write the XML files. This library
allows us to use stax in a more user-friendly
manner\footnote{\url{http://staxmate.codehaus.org}}.
\item[xstream-1.3.1.jar] Used to read and write the XML files. This parser is
used to parse the result from the SBML validator, JSBML might use it more
intensively in the future or drop it to use only
stax/woodstox\footnote{\url{http://xstream.codehaus.org}}.
\item[jigsaw-dateParser.jar] This is a stripped down version of the
jigsaw-library, containing one class to manipulate dates. It has been created
with the version from 2010-12-16\footnote{\url{http://jigsaw.w3.org}}.
\item[log4j-1.2.8.jar] JSBML uses the Apache log4j
logger\footnote{\url{http://logging.apache.org/log4j/}}. If you want to use
logging, you should include this logger.
\end{description}
JSBML was developed and tested with these versions of the libraries described
above. Some more recent versions might work, too. When you have all of these
dependencies in your build or class path alongside the JSBML JAR file, you are
ready to work with JSBML.


\subsection{Download and usage of the source distribution}

As an alternative to using the JAR files, you can check out the source tree from
SVN and compile JSBML yourself. To do that, you will need to have a Java
JDK\footref{fn:jvmldl} installed, the Apache
Ant\footnote{\url{http://ant.apache.org/}\label{fn:ant}} build system, and
Subversion\footnote{\url{http://subversion.apache.org/}\label{fn:svn}}, a
version control system.

%# to get the source
\begin{lstlisting}[language=bash,numbers=none,captionpos=t,
title={Use the following command to download the latest JSBML classes (requires
Subversion\footref{fn:svn}):}]
svn co "https://jsbml.svn.sourceforge.net/svnroot/jsbml/trunk" jsbml
cd jsbml
\end{lstlisting}

%# to compile and create the library JAR file
\begin{lstlisting}[language=bash,numbers=none,captionpos=t,
title={To compile the JSBML library to a single JAR file, type the following
command (requires Apache Ant\footref{fn:ant}):}]
ant jar
\end{lstlisting}

% (requires Apache Ant\footref{fn:ant})
%# to run the tests
\begin{lstlisting}[language=bash,numbers=none,captionpos=t,
title={If you want to run the JUnit\footref{fn:junit} tests
on your compiled JAR file, please use the following command:}]
ant test
\end{lstlisting}

If you performed all the steps above, you should have a JSBML library, based on
the latest version of all classes. You can now include the created JAR file into
your build or class path and start using JSBML.

\subsection{Download and usage of the JSBML modules}

JSBML provides today, two additional modules. Binary versions of the modules can
be found at the download site of JSBML. In order to obtain the most recent
version of the modules, please type the following Subversion\footref{fn:svn}
commands on your command line.

\begin{lstlisting}[language=bash,numbers=none,captionpos=t,
title={The CellDesigner bridge module should help CellDesigner plugin developers
to use JSBML as internal data structure.}]
svn co "https://jsbml.svn.sourceforge.net/svnroot/jsbml/modules/cellDesigner" cellDesigner
\end{lstlisting}

\begin{lstlisting}[language=bash,numbers=none,captionpos=t,
title={Developers, who still want to make use of libSBML, might want to have a
look at the libSBML communication layer.}]
svn co "https://jsbml.svn.sourceforge.net/svnroot/jsbml/modules/libSBMLio/" libSBMLio
\end{lstlisting}

% There is not module yet, so nothing the checkout. 
% \begin{lstlisting}[language=bash,numbers=none,captionpos=t,
% title={The third module is a compatibility module to ease switching from libSBML
% to JSBML.}]
% svn co "https://jsbml.svn.sourceforge.net/svnroot/jsbml/modules/libSBMLcompat" libSBMLcompat
% \end{lstlisting}

%\noindent All these modules can be compiled and included into your project in
% the same way as described for the JSBML main library.
