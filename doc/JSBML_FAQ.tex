For questions regarding SBML, please see the SBML FAQ at
\url{http://sbml.org/Documents/FAQ}.
\begin{description}
\item[Why does the class \texttt{LocalParameter} not inherit from
\texttt{Parameter}?]
\index{parameter!\texttt{LocalParameter}}
\index{parameter!\texttt{Parameter}}
The reason is the Boolean
\index{Boolean}
attribute \texttt{constant}, which is present in
\index{constant}
\index{parameter!\texttt{constant}}
\texttt{Parameter} and can be set to \texttt{false}. A parameter in the meaning
of SBML is not a constant, it might be some system variable
\index{JSBML!variable@\texttt{Variable}}
and can therefore be the subject of \texttt{Rule}s,
\index{rule}
\texttt{Event}s\index{event!\texttt{Event}}, \texttt{InitialAssignment}s
\index{InitialAssignment@\texttt{InitialAssignment}}
and so on, i.e., all instances of \texttt{Assignment},
\index{JSBML!assignment@\texttt{Assignment}}
whereas a \texttt{LocalParameter} is defined as a constant quantity that never
changes its value during the evaluation of a model\index{model}. It would
therefore only be possible to let \texttt{Parameter} inherit from
\texttt{LocalParameter} but this could lead to a semantic misinterpretation.


\item[Does JSBML depend on SWING or any particular graphical user interface
implementation?]
Although all classes in JSBML implement the \texttt{TreeNode} interface, which
is located in the package \texttt{javax.swing.tree}, all classes in JSBML are
entirely independent from any graphical user interface, such as the SWING
implementation. When loading the \texttt{TreeNode} interface, no other class
from SWING will be initialized or loaded; hence JSBML can also be used on
computers that do not provide any graphical system without the necessity of
catching a \texttt{HeadlessException}. The \texttt{TreeNode} interface only
defines methods and properties that all recursive tree data structures have to
implement anyway. Letting JSBML classes extend this interface makes JSBML
compatible with many other Java classes and methods that make use of the
standard \texttt{TreeNode} interface, hence ensuring a high compatibility with
other Java libraries. Since the SWING package belongs to the standard
Java\texttrademark{} distribution, it is ensured that the \texttt{TreeNode}
interface can always be localized by the Java Virtual Machine,
independent from the specific hardware or system.

\item[Does the usuage of the the \texttt{java.beens} package for the
\texttt{TreeNodeChangeListener} lead to an incompatibility with light-weight
Java installations?]
With the \texttt{java.beens} package being part of the standard Java
distribution, such an incompatibility will not occur. Extending existing
standard Java classes leads to a higher compatibility with other libraries and
should therefore be the preferred way to go in the development of JSBML.
\end{description}
